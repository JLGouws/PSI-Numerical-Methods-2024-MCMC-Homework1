\documentclass[11pt]{article}

    \usepackage[breakable]{tcolorbox}
    \usepackage{parskip} % Stop auto-indenting (to mimic markdown behaviour)
    

    % Basic figure setup, for now with no caption control since it's done
    % automatically by Pandoc (which extracts ![](path) syntax from Markdown).
    \usepackage{graphicx}
    % Maintain compatibility with old templates. Remove in nbconvert 6.0
    \let\Oldincludegraphics\includegraphics
    % Ensure that by default, figures have no caption (until we provide a
    % proper Figure object with a Caption API and a way to capture that
    % in the conversion process - todo).
    \usepackage{caption}
    \DeclareCaptionFormat{nocaption}{}
    \captionsetup{format=nocaption,aboveskip=0pt,belowskip=0pt}

    \usepackage{float}
    \floatplacement{figure}{H} % forces figures to be placed at the correct location
    \usepackage{xcolor} % Allow colors to be defined
    \usepackage{enumerate} % Needed for markdown enumerations to work
    \usepackage{geometry} % Used to adjust the document margins
    \usepackage{amsmath} % Equations
    \usepackage{amssymb} % Equations
    \usepackage{textcomp} % defines textquotesingle
    % Hack from http://tex.stackexchange.com/a/47451/13684:
    \AtBeginDocument{%
        \def\PYZsq{\textquotesingle}% Upright quotes in Pygmentized code
    }
    \usepackage{upquote} % Upright quotes for verbatim code
    \usepackage{eurosym} % defines \euro

    \usepackage{iftex}
    \ifPDFTeX
        \usepackage[T1]{fontenc}
        \IfFileExists{alphabeta.sty}{
              \usepackage{alphabeta}
          }{
              \usepackage[mathletters]{ucs}
              \usepackage[utf8x]{inputenc}
          }
    \else
        \usepackage{fontspec}
        \usepackage{unicode-math}
    \fi

    \usepackage{fancyvrb} % verbatim replacement that allows latex
    \usepackage{grffile} % extends the file name processing of package graphics
                         % to support a larger range
    \makeatletter % fix for old versions of grffile with XeLaTeX
    \@ifpackagelater{grffile}{2019/11/01}
    {
      % Do nothing on new versions
    }
    {
      \def\Gread@@xetex#1{%
        \IfFileExists{"\Gin@base".bb}%
        {\Gread@eps{\Gin@base.bb}}%
        {\Gread@@xetex@aux#1}%
      }
    }
    \makeatother
    \usepackage[Export]{adjustbox} % Used to constrain images to a maximum size
    \adjustboxset{max size={0.9\linewidth}{0.9\paperheight}}

    % The hyperref package gives us a pdf with properly built
    % internal navigation ('pdf bookmarks' for the table of contents,
    % internal cross-reference links, web links for URLs, etc.)
    \usepackage{hyperref}
    % The default LaTeX title has an obnoxious amount of whitespace. By default,
    % titling removes some of it. It also provides customization options.
    \usepackage{titling}
    \usepackage{longtable} % longtable support required by pandoc >1.10
    \usepackage{booktabs}  % table support for pandoc > 1.12.2
    \usepackage{array}     % table support for pandoc >= 2.11.3
    \usepackage{calc}      % table minipage width calculation for pandoc >= 2.11.1
    \usepackage[inline]{enumitem} % IRkernel/repr support (it uses the enumerate* environment)
    \usepackage[normalem]{ulem} % ulem is needed to support strikethroughs (\sout)
                                % normalem makes italics be italics, not underlines
    \usepackage{soul}      % strikethrough (\st) support for pandoc >= 3.0.0
    \usepackage{mathrsfs}
    

    
    % Colors for the hyperref package
    \definecolor{urlcolor}{rgb}{0,.145,.698}
    \definecolor{linkcolor}{rgb}{.71,0.21,0.01}
    \definecolor{citecolor}{rgb}{.12,.54,.11}

    % ANSI colors
    \definecolor{ansi-black}{HTML}{3E424D}
    \definecolor{ansi-black-intense}{HTML}{282C36}
    \definecolor{ansi-red}{HTML}{E75C58}
    \definecolor{ansi-red-intense}{HTML}{B22B31}
    \definecolor{ansi-green}{HTML}{00A250}
    \definecolor{ansi-green-intense}{HTML}{007427}
    \definecolor{ansi-yellow}{HTML}{DDB62B}
    \definecolor{ansi-yellow-intense}{HTML}{B27D12}
    \definecolor{ansi-blue}{HTML}{208FFB}
    \definecolor{ansi-blue-intense}{HTML}{0065CA}
    \definecolor{ansi-magenta}{HTML}{D160C4}
    \definecolor{ansi-magenta-intense}{HTML}{A03196}
    \definecolor{ansi-cyan}{HTML}{60C6C8}
    \definecolor{ansi-cyan-intense}{HTML}{258F8F}
    \definecolor{ansi-white}{HTML}{C5C1B4}
    \definecolor{ansi-white-intense}{HTML}{A1A6B2}
    \definecolor{ansi-default-inverse-fg}{HTML}{FFFFFF}
    \definecolor{ansi-default-inverse-bg}{HTML}{000000}

    % common color for the border for error outputs.
    \definecolor{outerrorbackground}{HTML}{FFDFDF}

    % commands and environments needed by pandoc snippets
    % extracted from the output of `pandoc -s`
    \providecommand{\tightlist}{%
      \setlength{\itemsep}{0pt}\setlength{\parskip}{0pt}}
    \DefineVerbatimEnvironment{Highlighting}{Verbatim}{commandchars=\\\{\}}
    % Add ',fontsize=\small' for more characters per line
    \newenvironment{Shaded}{}{}
    \newcommand{\KeywordTok}[1]{\textcolor[rgb]{0.00,0.44,0.13}{\textbf{{#1}}}}
    \newcommand{\DataTypeTok}[1]{\textcolor[rgb]{0.56,0.13,0.00}{{#1}}}
    \newcommand{\DecValTok}[1]{\textcolor[rgb]{0.25,0.63,0.44}{{#1}}}
    \newcommand{\BaseNTok}[1]{\textcolor[rgb]{0.25,0.63,0.44}{{#1}}}
    \newcommand{\FloatTok}[1]{\textcolor[rgb]{0.25,0.63,0.44}{{#1}}}
    \newcommand{\CharTok}[1]{\textcolor[rgb]{0.25,0.44,0.63}{{#1}}}
    \newcommand{\StringTok}[1]{\textcolor[rgb]{0.25,0.44,0.63}{{#1}}}
    \newcommand{\CommentTok}[1]{\textcolor[rgb]{0.38,0.63,0.69}{\textit{{#1}}}}
    \newcommand{\OtherTok}[1]{\textcolor[rgb]{0.00,0.44,0.13}{{#1}}}
    \newcommand{\AlertTok}[1]{\textcolor[rgb]{1.00,0.00,0.00}{\textbf{{#1}}}}
    \newcommand{\FunctionTok}[1]{\textcolor[rgb]{0.02,0.16,0.49}{{#1}}}
    \newcommand{\RegionMarkerTok}[1]{{#1}}
    \newcommand{\ErrorTok}[1]{\textcolor[rgb]{1.00,0.00,0.00}{\textbf{{#1}}}}
    \newcommand{\NormalTok}[1]{{#1}}

    % Additional commands for more recent versions of Pandoc
    \newcommand{\ConstantTok}[1]{\textcolor[rgb]{0.53,0.00,0.00}{{#1}}}
    \newcommand{\SpecialCharTok}[1]{\textcolor[rgb]{0.25,0.44,0.63}{{#1}}}
    \newcommand{\VerbatimStringTok}[1]{\textcolor[rgb]{0.25,0.44,0.63}{{#1}}}
    \newcommand{\SpecialStringTok}[1]{\textcolor[rgb]{0.73,0.40,0.53}{{#1}}}
    \newcommand{\ImportTok}[1]{{#1}}
    \newcommand{\DocumentationTok}[1]{\textcolor[rgb]{0.73,0.13,0.13}{\textit{{#1}}}}
    \newcommand{\AnnotationTok}[1]{\textcolor[rgb]{0.38,0.63,0.69}{\textbf{\textit{{#1}}}}}
    \newcommand{\CommentVarTok}[1]{\textcolor[rgb]{0.38,0.63,0.69}{\textbf{\textit{{#1}}}}}
    \newcommand{\VariableTok}[1]{\textcolor[rgb]{0.10,0.09,0.49}{{#1}}}
    \newcommand{\ControlFlowTok}[1]{\textcolor[rgb]{0.00,0.44,0.13}{\textbf{{#1}}}}
    \newcommand{\OperatorTok}[1]{\textcolor[rgb]{0.40,0.40,0.40}{{#1}}}
    \newcommand{\BuiltInTok}[1]{{#1}}
    \newcommand{\ExtensionTok}[1]{{#1}}
    \newcommand{\PreprocessorTok}[1]{\textcolor[rgb]{0.74,0.48,0.00}{{#1}}}
    \newcommand{\AttributeTok}[1]{\textcolor[rgb]{0.49,0.56,0.16}{{#1}}}
    \newcommand{\InformationTok}[1]{\textcolor[rgb]{0.38,0.63,0.69}{\textbf{\textit{{#1}}}}}
    \newcommand{\WarningTok}[1]{\textcolor[rgb]{0.38,0.63,0.69}{\textbf{\textit{{#1}}}}}


    % Define a nice break command that doesn't care if a line doesn't already
    % exist.
    \def\br{\hspace*{\fill} \\* }
    % Math Jax compatibility definitions
    \def\gt{>}
    \def\lt{<}
    \let\Oldtex\TeX
    \let\Oldlatex\LaTeX
    \renewcommand{\TeX}{\textrm{\Oldtex}}
    \renewcommand{\LaTeX}{\textrm{\Oldlatex}}
    % Document parameters
    % Document title
    \title{MCMC-homework-assignment}
    
    
    
    
    
    
    
% Pygments definitions
\makeatletter
\def\PY@reset{\let\PY@it=\relax \let\PY@bf=\relax%
    \let\PY@ul=\relax \let\PY@tc=\relax%
    \let\PY@bc=\relax \let\PY@ff=\relax}
\def\PY@tok#1{\csname PY@tok@#1\endcsname}
\def\PY@toks#1+{\ifx\relax#1\empty\else%
    \PY@tok{#1}\expandafter\PY@toks\fi}
\def\PY@do#1{\PY@bc{\PY@tc{\PY@ul{%
    \PY@it{\PY@bf{\PY@ff{#1}}}}}}}
\def\PY#1#2{\PY@reset\PY@toks#1+\relax+\PY@do{#2}}

\@namedef{PY@tok@w}{\def\PY@tc##1{\textcolor[rgb]{0.73,0.73,0.73}{##1}}}
\@namedef{PY@tok@c}{\let\PY@it=\textit\def\PY@tc##1{\textcolor[rgb]{0.24,0.48,0.48}{##1}}}
\@namedef{PY@tok@cp}{\def\PY@tc##1{\textcolor[rgb]{0.61,0.40,0.00}{##1}}}
\@namedef{PY@tok@k}{\let\PY@bf=\textbf\def\PY@tc##1{\textcolor[rgb]{0.00,0.50,0.00}{##1}}}
\@namedef{PY@tok@kp}{\def\PY@tc##1{\textcolor[rgb]{0.00,0.50,0.00}{##1}}}
\@namedef{PY@tok@kt}{\def\PY@tc##1{\textcolor[rgb]{0.69,0.00,0.25}{##1}}}
\@namedef{PY@tok@o}{\def\PY@tc##1{\textcolor[rgb]{0.40,0.40,0.40}{##1}}}
\@namedef{PY@tok@ow}{\let\PY@bf=\textbf\def\PY@tc##1{\textcolor[rgb]{0.67,0.13,1.00}{##1}}}
\@namedef{PY@tok@nb}{\def\PY@tc##1{\textcolor[rgb]{0.00,0.50,0.00}{##1}}}
\@namedef{PY@tok@nf}{\def\PY@tc##1{\textcolor[rgb]{0.00,0.00,1.00}{##1}}}
\@namedef{PY@tok@nc}{\let\PY@bf=\textbf\def\PY@tc##1{\textcolor[rgb]{0.00,0.00,1.00}{##1}}}
\@namedef{PY@tok@nn}{\let\PY@bf=\textbf\def\PY@tc##1{\textcolor[rgb]{0.00,0.00,1.00}{##1}}}
\@namedef{PY@tok@ne}{\let\PY@bf=\textbf\def\PY@tc##1{\textcolor[rgb]{0.80,0.25,0.22}{##1}}}
\@namedef{PY@tok@nv}{\def\PY@tc##1{\textcolor[rgb]{0.10,0.09,0.49}{##1}}}
\@namedef{PY@tok@no}{\def\PY@tc##1{\textcolor[rgb]{0.53,0.00,0.00}{##1}}}
\@namedef{PY@tok@nl}{\def\PY@tc##1{\textcolor[rgb]{0.46,0.46,0.00}{##1}}}
\@namedef{PY@tok@ni}{\let\PY@bf=\textbf\def\PY@tc##1{\textcolor[rgb]{0.44,0.44,0.44}{##1}}}
\@namedef{PY@tok@na}{\def\PY@tc##1{\textcolor[rgb]{0.41,0.47,0.13}{##1}}}
\@namedef{PY@tok@nt}{\let\PY@bf=\textbf\def\PY@tc##1{\textcolor[rgb]{0.00,0.50,0.00}{##1}}}
\@namedef{PY@tok@nd}{\def\PY@tc##1{\textcolor[rgb]{0.67,0.13,1.00}{##1}}}
\@namedef{PY@tok@s}{\def\PY@tc##1{\textcolor[rgb]{0.73,0.13,0.13}{##1}}}
\@namedef{PY@tok@sd}{\let\PY@it=\textit\def\PY@tc##1{\textcolor[rgb]{0.73,0.13,0.13}{##1}}}
\@namedef{PY@tok@si}{\let\PY@bf=\textbf\def\PY@tc##1{\textcolor[rgb]{0.64,0.35,0.47}{##1}}}
\@namedef{PY@tok@se}{\let\PY@bf=\textbf\def\PY@tc##1{\textcolor[rgb]{0.67,0.36,0.12}{##1}}}
\@namedef{PY@tok@sr}{\def\PY@tc##1{\textcolor[rgb]{0.64,0.35,0.47}{##1}}}
\@namedef{PY@tok@ss}{\def\PY@tc##1{\textcolor[rgb]{0.10,0.09,0.49}{##1}}}
\@namedef{PY@tok@sx}{\def\PY@tc##1{\textcolor[rgb]{0.00,0.50,0.00}{##1}}}
\@namedef{PY@tok@m}{\def\PY@tc##1{\textcolor[rgb]{0.40,0.40,0.40}{##1}}}
\@namedef{PY@tok@gh}{\let\PY@bf=\textbf\def\PY@tc##1{\textcolor[rgb]{0.00,0.00,0.50}{##1}}}
\@namedef{PY@tok@gu}{\let\PY@bf=\textbf\def\PY@tc##1{\textcolor[rgb]{0.50,0.00,0.50}{##1}}}
\@namedef{PY@tok@gd}{\def\PY@tc##1{\textcolor[rgb]{0.63,0.00,0.00}{##1}}}
\@namedef{PY@tok@gi}{\def\PY@tc##1{\textcolor[rgb]{0.00,0.52,0.00}{##1}}}
\@namedef{PY@tok@gr}{\def\PY@tc##1{\textcolor[rgb]{0.89,0.00,0.00}{##1}}}
\@namedef{PY@tok@ge}{\let\PY@it=\textit}
\@namedef{PY@tok@gs}{\let\PY@bf=\textbf}
\@namedef{PY@tok@ges}{\let\PY@bf=\textbf\let\PY@it=\textit}
\@namedef{PY@tok@gp}{\let\PY@bf=\textbf\def\PY@tc##1{\textcolor[rgb]{0.00,0.00,0.50}{##1}}}
\@namedef{PY@tok@go}{\def\PY@tc##1{\textcolor[rgb]{0.44,0.44,0.44}{##1}}}
\@namedef{PY@tok@gt}{\def\PY@tc##1{\textcolor[rgb]{0.00,0.27,0.87}{##1}}}
\@namedef{PY@tok@err}{\def\PY@bc##1{{\setlength{\fboxsep}{\string -\fboxrule}\fcolorbox[rgb]{1.00,0.00,0.00}{1,1,1}{\strut ##1}}}}
\@namedef{PY@tok@kc}{\let\PY@bf=\textbf\def\PY@tc##1{\textcolor[rgb]{0.00,0.50,0.00}{##1}}}
\@namedef{PY@tok@kd}{\let\PY@bf=\textbf\def\PY@tc##1{\textcolor[rgb]{0.00,0.50,0.00}{##1}}}
\@namedef{PY@tok@kn}{\let\PY@bf=\textbf\def\PY@tc##1{\textcolor[rgb]{0.00,0.50,0.00}{##1}}}
\@namedef{PY@tok@kr}{\let\PY@bf=\textbf\def\PY@tc##1{\textcolor[rgb]{0.00,0.50,0.00}{##1}}}
\@namedef{PY@tok@bp}{\def\PY@tc##1{\textcolor[rgb]{0.00,0.50,0.00}{##1}}}
\@namedef{PY@tok@fm}{\def\PY@tc##1{\textcolor[rgb]{0.00,0.00,1.00}{##1}}}
\@namedef{PY@tok@vc}{\def\PY@tc##1{\textcolor[rgb]{0.10,0.09,0.49}{##1}}}
\@namedef{PY@tok@vg}{\def\PY@tc##1{\textcolor[rgb]{0.10,0.09,0.49}{##1}}}
\@namedef{PY@tok@vi}{\def\PY@tc##1{\textcolor[rgb]{0.10,0.09,0.49}{##1}}}
\@namedef{PY@tok@vm}{\def\PY@tc##1{\textcolor[rgb]{0.10,0.09,0.49}{##1}}}
\@namedef{PY@tok@sa}{\def\PY@tc##1{\textcolor[rgb]{0.73,0.13,0.13}{##1}}}
\@namedef{PY@tok@sb}{\def\PY@tc##1{\textcolor[rgb]{0.73,0.13,0.13}{##1}}}
\@namedef{PY@tok@sc}{\def\PY@tc##1{\textcolor[rgb]{0.73,0.13,0.13}{##1}}}
\@namedef{PY@tok@dl}{\def\PY@tc##1{\textcolor[rgb]{0.73,0.13,0.13}{##1}}}
\@namedef{PY@tok@s2}{\def\PY@tc##1{\textcolor[rgb]{0.73,0.13,0.13}{##1}}}
\@namedef{PY@tok@sh}{\def\PY@tc##1{\textcolor[rgb]{0.73,0.13,0.13}{##1}}}
\@namedef{PY@tok@s1}{\def\PY@tc##1{\textcolor[rgb]{0.73,0.13,0.13}{##1}}}
\@namedef{PY@tok@mb}{\def\PY@tc##1{\textcolor[rgb]{0.40,0.40,0.40}{##1}}}
\@namedef{PY@tok@mf}{\def\PY@tc##1{\textcolor[rgb]{0.40,0.40,0.40}{##1}}}
\@namedef{PY@tok@mh}{\def\PY@tc##1{\textcolor[rgb]{0.40,0.40,0.40}{##1}}}
\@namedef{PY@tok@mi}{\def\PY@tc##1{\textcolor[rgb]{0.40,0.40,0.40}{##1}}}
\@namedef{PY@tok@il}{\def\PY@tc##1{\textcolor[rgb]{0.40,0.40,0.40}{##1}}}
\@namedef{PY@tok@mo}{\def\PY@tc##1{\textcolor[rgb]{0.40,0.40,0.40}{##1}}}
\@namedef{PY@tok@ch}{\let\PY@it=\textit\def\PY@tc##1{\textcolor[rgb]{0.24,0.48,0.48}{##1}}}
\@namedef{PY@tok@cm}{\let\PY@it=\textit\def\PY@tc##1{\textcolor[rgb]{0.24,0.48,0.48}{##1}}}
\@namedef{PY@tok@cpf}{\let\PY@it=\textit\def\PY@tc##1{\textcolor[rgb]{0.24,0.48,0.48}{##1}}}
\@namedef{PY@tok@c1}{\let\PY@it=\textit\def\PY@tc##1{\textcolor[rgb]{0.24,0.48,0.48}{##1}}}
\@namedef{PY@tok@cs}{\let\PY@it=\textit\def\PY@tc##1{\textcolor[rgb]{0.24,0.48,0.48}{##1}}}

\def\PYZbs{\char`\\}
\def\PYZus{\char`\_}
\def\PYZob{\char`\{}
\def\PYZcb{\char`\}}
\def\PYZca{\char`\^}
\def\PYZam{\char`\&}
\def\PYZlt{\char`\<}
\def\PYZgt{\char`\>}
\def\PYZsh{\char`\#}
\def\PYZpc{\char`\%}
\def\PYZdl{\char`\$}
\def\PYZhy{\char`\-}
\def\PYZsq{\char`\'}
\def\PYZdq{\char`\"}
\def\PYZti{\char`\~}
% for compatibility with earlier versions
\def\PYZat{@}
\def\PYZlb{[}
\def\PYZrb{]}
\makeatother


    % For linebreaks inside Verbatim environment from package fancyvrb.
    \makeatletter
        \newbox\Wrappedcontinuationbox
        \newbox\Wrappedvisiblespacebox
        \newcommand*\Wrappedvisiblespace {\textcolor{red}{\textvisiblespace}}
        \newcommand*\Wrappedcontinuationsymbol {\textcolor{red}{\llap{\tiny$\m@th\hookrightarrow$}}}
        \newcommand*\Wrappedcontinuationindent {3ex }
        \newcommand*\Wrappedafterbreak {\kern\Wrappedcontinuationindent\copy\Wrappedcontinuationbox}
        % Take advantage of the already applied Pygments mark-up to insert
        % potential linebreaks for TeX processing.
        %        {, <, #, %, $, ' and ": go to next line.
        %        _, }, ^, &, >, - and ~: stay at end of broken line.
        % Use of \textquotesingle for straight quote.
        \newcommand*\Wrappedbreaksatspecials {%
            \def\PYGZus{\discretionary{\char`\_}{\Wrappedafterbreak}{\char`\_}}%
            \def\PYGZob{\discretionary{}{\Wrappedafterbreak\char`\{}{\char`\{}}%
            \def\PYGZcb{\discretionary{\char`\}}{\Wrappedafterbreak}{\char`\}}}%
            \def\PYGZca{\discretionary{\char`\^}{\Wrappedafterbreak}{\char`\^}}%
            \def\PYGZam{\discretionary{\char`\&}{\Wrappedafterbreak}{\char`\&}}%
            \def\PYGZlt{\discretionary{}{\Wrappedafterbreak\char`\<}{\char`\<}}%
            \def\PYGZgt{\discretionary{\char`\>}{\Wrappedafterbreak}{\char`\>}}%
            \def\PYGZsh{\discretionary{}{\Wrappedafterbreak\char`\#}{\char`\#}}%
            \def\PYGZpc{\discretionary{}{\Wrappedafterbreak\char`\%}{\char`\%}}%
            \def\PYGZdl{\discretionary{}{\Wrappedafterbreak\char`\$}{\char`\$}}%
            \def\PYGZhy{\discretionary{\char`\-}{\Wrappedafterbreak}{\char`\-}}%
            \def\PYGZsq{\discretionary{}{\Wrappedafterbreak\textquotesingle}{\textquotesingle}}%
            \def\PYGZdq{\discretionary{}{\Wrappedafterbreak\char`\"}{\char`\"}}%
            \def\PYGZti{\discretionary{\char`\~}{\Wrappedafterbreak}{\char`\~}}%
        }
        % Some characters . , ; ? ! / are not pygmentized.
        % This macro makes them "active" and they will insert potential linebreaks
        \newcommand*\Wrappedbreaksatpunct {%
            \lccode`\~`\.\lowercase{\def~}{\discretionary{\hbox{\char`\.}}{\Wrappedafterbreak}{\hbox{\char`\.}}}%
            \lccode`\~`\,\lowercase{\def~}{\discretionary{\hbox{\char`\,}}{\Wrappedafterbreak}{\hbox{\char`\,}}}%
            \lccode`\~`\;\lowercase{\def~}{\discretionary{\hbox{\char`\;}}{\Wrappedafterbreak}{\hbox{\char`\;}}}%
            \lccode`\~`\:\lowercase{\def~}{\discretionary{\hbox{\char`\:}}{\Wrappedafterbreak}{\hbox{\char`\:}}}%
            \lccode`\~`\?\lowercase{\def~}{\discretionary{\hbox{\char`\?}}{\Wrappedafterbreak}{\hbox{\char`\?}}}%
            \lccode`\~`\!\lowercase{\def~}{\discretionary{\hbox{\char`\!}}{\Wrappedafterbreak}{\hbox{\char`\!}}}%
            \lccode`\~`\/\lowercase{\def~}{\discretionary{\hbox{\char`\/}}{\Wrappedafterbreak}{\hbox{\char`\/}}}%
            \catcode`\.\active
            \catcode`\,\active
            \catcode`\;\active
            \catcode`\:\active
            \catcode`\?\active
            \catcode`\!\active
            \catcode`\/\active
            \lccode`\~`\~
        }
    \makeatother

    \let\OriginalVerbatim=\Verbatim
    \makeatletter
    \renewcommand{\Verbatim}[1][1]{%
        %\parskip\z@skip
        \sbox\Wrappedcontinuationbox {\Wrappedcontinuationsymbol}%
        \sbox\Wrappedvisiblespacebox {\FV@SetupFont\Wrappedvisiblespace}%
        \def\FancyVerbFormatLine ##1{\hsize\linewidth
            \vtop{\raggedright\hyphenpenalty\z@\exhyphenpenalty\z@
                \doublehyphendemerits\z@\finalhyphendemerits\z@
                \strut ##1\strut}%
        }%
        % If the linebreak is at a space, the latter will be displayed as visible
        % space at end of first line, and a continuation symbol starts next line.
        % Stretch/shrink are however usually zero for typewriter font.
        \def\FV@Space {%
            \nobreak\hskip\z@ plus\fontdimen3\font minus\fontdimen4\font
            \discretionary{\copy\Wrappedvisiblespacebox}{\Wrappedafterbreak}
            {\kern\fontdimen2\font}%
        }%

        % Allow breaks at special characters using \PYG... macros.
        \Wrappedbreaksatspecials
        % Breaks at punctuation characters . , ; ? ! and / need catcode=\active
        \OriginalVerbatim[#1,codes*=\Wrappedbreaksatpunct]%
    }
    \makeatother

    % Exact colors from NB
    \definecolor{incolor}{HTML}{303F9F}
    \definecolor{outcolor}{HTML}{D84315}
    \definecolor{cellborder}{HTML}{CFCFCF}
    \definecolor{cellbackground}{HTML}{F7F7F7}

    % prompt
    \makeatletter
    \newcommand{\boxspacing}{\kern\kvtcb@left@rule\kern\kvtcb@boxsep}
    \makeatother
    \newcommand{\prompt}[4]{
        {\ttfamily\llap{{\color{#2}[#3]:\hspace{3pt}#4}}\vspace{-\baselineskip}}
    }
    

    
    % Prevent overflowing lines due to hard-to-break entities
    \sloppy
    % Setup hyperref package
    \hypersetup{
      breaklinks=true,  % so long urls are correctly broken across lines
      colorlinks=true,
      urlcolor=urlcolor,
      linkcolor=linkcolor,
      citecolor=citecolor,
      }
    % Slightly bigger margins than the latex defaults
    
    \geometry{verbose,tmargin=1in,bmargin=1in,lmargin=1in,rmargin=1in}
    
    

\begin{document}
    
    \maketitle
    
    

    
    \hypertarget{psi-numerical-methods-2024---homework-assignment-on-model-fitting-mcmc}{%
\section{PSI Numerical Methods 2024 - Homework Assignment on Model
Fitting \&
MCMC}\label{psi-numerical-methods-2024---homework-assignment-on-model-fitting-mcmc}}

    \hypertarget{code-available-here}{%
\subsection{\texorpdfstring{Code available
\href{https://github.com/JLGouws/PSI-Numerical-Methods-2024-MCMC-Homework1}{here}}{Code available here}}\label{code-available-here}}

    We're going to put together everything we have learned so far to re-do
the data analysis for the Perlmutter et al.~1999 paper on the discovery
of dark energy!
(https://ui.adsabs.harvard.edu/abs/1999ApJ\ldots517..565P/abstract)

Start by \texttt{Forking} this repository on Github:
https://github.com/dstndstn/PSI-Numerical-Methods-2024-MCMC-Homework And
then clone the repository to your laptop or to Symmetry. You can modify
this notebook, and when you are done, save it, and then
\texttt{git\ commit\ -a} the results, and \texttt{git\ push} them back
to your fork of the repository. You will ``hand in'' your homework by
giving a link to your Github repository, where the marker will be able
to read your notebook.

First, a little bit of background on the cosmology and astrophysics. The
paper reports measurements of a group of supernova explosions of a
specific type, ``Type 1a''. These are thought to be caused by a white
dwarf star that has a companion star that ``donates'' gas to the white
dwarf. It gradually gains mass until it exceeds the Chandresekhar mass,
and explodes. Since they all explode through the same mechanism, and
with the same mass, they should all have the same intrinsic brightess.
It turns out to be a \emph{little} more complicated than that, but in
the end, these Type-1a supernovae can be turned into ``standard
candles'', objects that are all the same brightness. If you can also
measure the redshift of each galaxy containing the supernova, then you
can map out this brightness--redshift relation, and the shape of that
relation depends on how the universe grows over cosmic time. In turn,
the growth rate of the universe depends on the contents of the universe!

In this way, these Type-1a supernova allow us to constrain the
parameters of a model of the universe. Specifically, the model is called
``Lambda-CDM'', a universe containing dark energy and matter (cold dark
matter, plus regular matter). We will consider a two-parameter version
of this model: \(\Omega_M\), the amount of matter, and
\(\Omega_{\Lambda}\), the amount of dark energy. These are in cosmology
units of ``energy density now relative to the critical density'', where
the critical density is the energy density you need for the universe to
be spatially flat (angles of a large triangle sum to 180 degrees). So
\(\Omega_M = 1\), \(\Omega_{\Lambda} = 0\) would be a flat universe
containing all matter, while \(\Omega_M = 0.25\),
\(\Omega_{\Lambda} = 0.5\) would be a spatially closed universe with
dark energy and matter. Varying these ingredients changes the growth
history of the universe, which changes how much the light from a
supernova is redshifted, and how its brightness drops off with distance.

(In the code below, we will call these \texttt{Omega\_M} = \(\Omega_M\)
and \texttt{Omega\_DE} = \(\Omega_{\Lambda}\).)

Distance measurements in cosmology are complicated -- see
https://arxiv.org/abs/astro-ph/9905116 for details! For this assignment,
we will use a cosmology package that will handle all this for us. All we
need to use is the ``luminosity distance'', which is the one that tells
you how objects get fainter given a redshift.

    \begin{tcolorbox}[breakable, size=fbox, boxrule=1pt, pad at break*=1mm,colback=cellbackground, colframe=cellborder]
\prompt{In}{incolor}{1}{\boxspacing}
\begin{Verbatim}[commandchars=\\\{\}]
\PY{c}{\PYZsh{} Let\PYZsq{}s start by installing the Cosmology package!}
\PY{k}{using}\PY{+w}{ }\PY{n}{Pkg}
\PY{n}{Pkg}\PY{o}{.}\PY{n}{add}\PY{p}{(}\PY{l+s}{\PYZdq{}}\PY{l+s}{Cosmology}\PY{l+s}{\PYZdq{}}\PY{p}{)}
\end{Verbatim}
\end{tcolorbox}

    \begin{Verbatim}[commandchars=\\\{\}]
\textcolor{ansi-green-intense}{\textbf{   Resolving}} package versions{\ldots}
\textcolor{ansi-green-intense}{\textbf{  No Changes}} to `\textasciitilde{}/.julia/environments/v1.10/Project.toml`
\textcolor{ansi-green-intense}{\textbf{  No Changes}} to `\textasciitilde{}/.julia/environments/v1.10/Manifest.toml`
    \end{Verbatim}

    \begin{tcolorbox}[breakable, size=fbox, boxrule=1pt, pad at break*=1mm,colback=cellbackground, colframe=cellborder]
\prompt{In}{incolor}{2}{\boxspacing}
\begin{Verbatim}[commandchars=\\\{\}]
\PY{c}{\PYZsh{} We\PYZsq{}ll also end up using all our old friends:}
\PY{k}{using}\PY{+w}{ }\PY{n}{WGLMakie}
\PY{k}{using}\PY{+w}{ }\PY{n}{CSV}
\PY{k}{using}\PY{+w}{ }\PY{n}{DataFrames}
\PY{k}{using}\PY{+w}{ }\PY{n}{Cosmology}
\PY{k}{using}\PY{+w}{ }\PY{n}{Statistics}
\end{Verbatim}
\end{tcolorbox}

    \begin{tcolorbox}[breakable, size=fbox, boxrule=1pt, pad at break*=1mm,colback=cellbackground, colframe=cellborder]
\prompt{In}{incolor}{3}{\boxspacing}
\begin{Verbatim}[commandchars=\\\{\}]
\PY{c}{\PYZsh{} There is a data file in this directory, taken basically straight out of the Perlmutter+1999 paper.  We can read it with the CSV package.}
\PY{n}{data}\PY{+w}{ }\PY{o}{=}\PY{+w}{ }\PY{n}{CSV}\PY{o}{.}\PY{n}{read}\PY{p}{(}\PY{l+s}{\PYZdq{}}\PY{l+s}{p99\PYZhy{}data.txt}\PY{l+s}{\PYZdq{}}\PY{p}{,}\PY{+w}{ }\PY{n}{DataFrame}\PY{p}{,}\PY{+w}{ }\PY{n}{delim}\PY{o}{=}\PY{l+s}{\PYZdq{}}\PY{l+s}{ }\PY{l+s}{\PYZdq{}}\PY{p}{,}\PY{+w}{ }\PY{n}{ignorerepeated}\PY{o}{=}\PY{n+nb}{true}\PY{p}{)}\PY{p}{;}
\end{Verbatim}
\end{tcolorbox}

    \begin{tcolorbox}[breakable, size=fbox, boxrule=1pt, pad at break*=1mm,colback=cellbackground, colframe=cellborder]
\prompt{In}{incolor}{4}{\boxspacing}
\begin{Verbatim}[commandchars=\\\{\}]
\PY{c}{\PYZsh{} Make a copy of the data columns that we want to treat as the \PYZdq{}y\PYZdq{} measurements.}
\PY{c}{\PYZsh{} These are the measured brightnesses, and their Gaussian uncertainties (standard deviations).}
\PY{n}{data}\PY{o}{.}\PY{n}{mag}\PY{+w}{ }\PY{o}{=}\PY{+w}{ }\PY{n}{data}\PY{o}{.}\PY{n}{m\PYZus{}b\PYZus{}eff}
\PY{n}{data}\PY{o}{.}\PY{n}{sigma\PYZus{}mag}\PY{+w}{ }\PY{o}{=}\PY{+w}{ }\PY{n}{data}\PY{o}{.}\PY{n}{sigma\PYZus{}m\PYZus{}b\PYZus{}eff}\PY{p}{;}
\end{Verbatim}
\end{tcolorbox}

    \begin{tcolorbox}[breakable, size=fbox, boxrule=1pt, pad at break*=1mm,colback=cellbackground, colframe=cellborder]
\prompt{In}{incolor}{5}{\boxspacing}
\begin{Verbatim}[commandchars=\\\{\}]
\PY{n}{f}\PY{+w}{ }\PY{o}{=}\PY{+w}{ }\PY{n}{Figure}\PY{p}{(}\PY{p}{)}
\PY{n}{Axis}\PY{p}{(}\PY{n}{f}\PY{p}{[}\PY{l+m+mi}{1}\PY{p}{,}\PY{l+m+mi}{1}\PY{p}{]}\PY{p}{,}\PY{+w}{ }\PY{n}{title}\PY{o}{=}\PY{l+s}{\PYZdq{}}\PY{l+s}{Perlmutter+99 Supernovae}\PY{l+s}{\PYZdq{}}\PY{p}{,}\PY{+w}{ }\PY{n}{xlabel}\PY{o}{=}\PY{l+s}{\PYZdq{}}\PY{l+s}{Redshift z}\PY{l+s}{\PYZdq{}}\PY{p}{,}\PY{+w}{ }\PY{n}{ylabel}\PY{o}{=}\PY{l+s}{\PYZdq{}}\PY{l+s}{m\PYZus{}B}\PY{l+s}{\PYZdq{}}\PY{p}{)}
\PY{n}{errorbars!}\PY{p}{(}\PY{n}{data}\PY{o}{.}\PY{n}{z}\PY{p}{,}\PY{+w}{ }\PY{n}{data}\PY{o}{.}\PY{n}{mag}\PY{p}{,}\PY{+w}{ }\PY{n}{data}\PY{o}{.}\PY{n}{sigma\PYZus{}mag}\PY{p}{)}
\PY{n}{scatter!}\PY{p}{(}\PY{n}{data}\PY{o}{.}\PY{n}{z}\PY{p}{,}\PY{+w}{ }\PY{n}{data}\PY{o}{.}\PY{n}{mag}\PY{p}{,}\PY{+w}{ }\PY{n}{markersize}\PY{o}{=}\PY{l+m+mi}{5}\PY{p}{,}\PY{+w}{ }\PY{n}{color}\PY{o}{=}\PY{l+s+ss}{:maroon}\PY{p}{)}
\PY{n}{save}\PY{p}{(}\PY{l+s}{\PYZdq{}}\PY{l+s}{perlmutterPlot.png}\PY{l+s}{\PYZdq{}}\PY{p}{,}\PY{+w}{ }\PY{n}{f}\PY{p}{)}
\PY{n}{display}\PY{p}{(}\PY{n}{f}\PY{p}{)}
\end{Verbatim}
\end{tcolorbox}
    
    \begin{center}
      
      \includegraphics[]{perlmutterPlot.png}
    \end{center}

    
    \begin{Verbatim}[commandchars=\\\{\}]
Bonito.App(Bonito.var"\#8\#14"\{WGLMakie.var"\#20\#21"\{WGLMakie.Screen, Scene\}\}(WGLMakie.var"\#20\#21"\{WGLMakie.Screen, Scene\}(WGLMakie.Screen(
    framerate = 30.0,
    resize\_to = nothing,
    px\_per\_unit = automatic,
    scalefactor = automatic,
    session = nothing,
    three = Channel\{Bool\}(1) (empty),
    scene = Scene (600px, 450px):
  0 Plots
  1 Child Scene:
    └ Scene (600px, 450px),
), Scene (600px, 450px):
  0 Plots
  1 Child Scene:
    └ Scene (600px, 450px))), Base.RefValue\{Union\{Nothing, Bonito.Session\}\}(nothing), "Bonito App", false)
    \end{Verbatim}

    
    
            \begin{tcolorbox}[breakable, size=fbox, boxrule=.5pt, pad at break*=1mm, opacityfill=0]
\prompt{Out}{outcolor}{5}{\boxspacing}
\begin{Verbatim}[commandchars=\\\{\}]
WGLMakie.Screen(
    framerate = 30.0,
    resize\_to = nothing,
    px\_per\_unit = automatic,
    scalefactor = automatic,
    session = true,
    three = Channel\{Bool\}(1) (1 item available),
    scene = Scene (600px, 450px):
  0 Plots
  1 Child Scene:
    └ Scene (600px, 450px),
)
\end{Verbatim}
\end{tcolorbox}
        
    \begin{tcolorbox}[breakable, size=fbox, boxrule=1pt, pad at break*=1mm,colback=cellbackground, colframe=cellborder]
\prompt{In}{incolor}{6}{\boxspacing}
\begin{Verbatim}[commandchars=\\\{\}]
\PY{c}{\PYZsh{} Here is how we will use the \PYZdq{}cosmology\PYZdq{} package.  This will create a cosmology \PYZdq{}object\PYZdq{} with the parameters we pass in.}
\PY{c}{\PYZsh{} It does not take an Omega\PYZus{}Lambda parameter; instead, it takes Omega\PYZus{}Matter, and Omega\PYZus{}K (for \PYZdq{}curvature\PYZdq{}), where}
\PY{c}{\PYZsh{} Omega\PYZus{}K = 1. \PYZhy{} Omatter \PYZhy{} Olambda.  We will also pass in \PYZdq{}Tcmb=0\PYZdq{}, which tells it to ignore the effects of radiation.}

\PY{n}{universe}\PY{+w}{ }\PY{o}{=}\PY{+w}{ }\PY{n}{cosmology}\PY{p}{(}\PY{n}{OmegaK}\PY{o}{=}\PY{l+m+mf}{0.1}\PY{p}{,}\PY{+w}{ }\PY{n}{OmegaM}\PY{o}{=}\PY{l+m+mf}{0.4}\PY{p}{,}\PY{+w}{ }\PY{n}{Tcmb}\PY{o}{=}\PY{l+m+mi}{0}\PY{p}{)}
\PY{n+nd}{@show}\PY{+w}{ }\PY{n}{universe}
\PY{n+nd}{@show}\PY{+w}{ }\PY{n}{universe}\PY{o}{.}\PY{n}{Ω\PYZus{}Λ}\PY{p}{;}
\end{Verbatim}
\end{tcolorbox}

    \begin{Verbatim}[commandchars=\\\{\}]
universe = Cosmology.OpenLCDM\{Float64\}(0.69, 0.1, 0.5, 0.4, 0.0)
universe.Ω\_Λ = 0.5
    \end{Verbatim}

    \begin{tcolorbox}[breakable, size=fbox, boxrule=1pt, pad at break*=1mm,colback=cellbackground, colframe=cellborder]
\prompt{In}{incolor}{7}{\boxspacing}
\begin{Verbatim}[commandchars=\\\{\}]
\PY{c}{\PYZsh{} We can then pass that \PYZdq{}universe\PYZdq{} object to other functions to compute things about it.  Basically the only one you\PYZsq{}ll}
\PY{c}{\PYZsh{} need is this `distance\PYZus{}modulus`, which tell you, in \PYZus{}magnitudes\PYZus{}, how much fainter an object is at the given redshift,}
\PY{c}{\PYZsh{} versus how faint it would be if it were 10 parsecs away.}

\PY{k}{function}\PY{+w}{ }\PY{n}{distance\PYZus{}modulus}\PY{p}{(}\PY{n}{universe}\PY{p}{,}\PY{+w}{ }\PY{n}{z}\PY{p}{)}
\PY{+w}{    }\PY{n}{DL}\PY{+w}{ }\PY{o}{=}\PY{+w}{ }\PY{n}{luminosity\PYZus{}dist}\PY{p}{(}\PY{n}{universe}\PY{p}{,}\PY{+w}{ }\PY{n}{z}\PY{p}{)}
\PY{+w}{    }\PY{c}{\PYZsh{} DL is in Megaparsecs; the distance for absolute to observed mag is 10 pc.}
\PY{+w}{    }\PY{l+m+mf}{5.}\PY{+w}{ }\PY{o}{*}\PY{+w}{ }\PY{n}{log10}\PY{o}{.}\PY{p}{(}\PY{n}{DL}\PY{o}{.}\PY{n}{val}\PY{+w}{ }\PY{o}{*}\PY{+w}{ }\PY{l+m+mf}{1e6}\PY{+w}{ }\PY{o}{/}\PY{+w}{ }\PY{l+m+mf}{10.}\PY{p}{)}
\PY{k}{end}\PY{p}{;}
\end{Verbatim}
\end{tcolorbox}

    There is one more parameter to the model we will be fitting: \(M\), the
\emph{absolute magnitude} of the supernovae. This is a ``nuisance
parameter'' - a parameter that we have to fit for, but that we don't
really care about; it's basically a calibration of what the intrinsic
brightness of a supernova is. To start out, we will fix this value to a
constant, but later we will fit for it along with our Omegas.

The \emph{observed} brightness of a supernova will be its \emph{absolute
mag} plus its \emph{distance modulus}. The \emph{distance modulus}
depends on the redshift \emph{z} and our parameters Omega\_M and
Omega\_DE.

    \begin{tcolorbox}[breakable, size=fbox, boxrule=1pt, pad at break*=1mm,colback=cellbackground, colframe=cellborder]
\prompt{In}{incolor}{8}{\boxspacing}
\begin{Verbatim}[commandchars=\\\{\}]
\PY{c}{\PYZsh{} We\PYZsq{}ll cheat a bit and use a \PYZdq{}nominal\PYZdq{} cosmology with currently\PYZhy{}accepted values of Omega\PYZus{}M = 0.29, Omega\PYZus{}DE = 0.71.}
\PY{n}{nominal}\PY{+w}{ }\PY{o}{=}\PY{+w}{ }\PY{n}{cosmology}\PY{p}{(}\PY{n}{Tcmb}\PY{o}{=}\PY{l+m+mi}{0}\PY{p}{)}

\PY{n}{f}\PY{+w}{ }\PY{o}{=}\PY{+w}{ }\PY{n}{Figure}\PY{p}{(}\PY{p}{)}
\PY{n}{ax}\PY{+w}{ }\PY{o}{=}\PY{+w}{ }\PY{n}{Axis}\PY{p}{(}\PY{n}{f}\PY{p}{[}\PY{l+m+mi}{1}\PY{p}{,}\PY{l+m+mi}{1}\PY{p}{]}\PY{p}{,}\PY{+w}{ }\PY{n}{title}\PY{o}{=}\PY{l+s}{\PYZdq{}}\PY{l+s}{Perlmutter+99 Supernovae}\PY{l+s}{\PYZdq{}}\PY{p}{,}\PY{+w}{ }\PY{n}{xlabel}\PY{o}{=}\PY{l+s}{\PYZdq{}}\PY{l+s}{Redshift z}\PY{l+s}{\PYZdq{}}\PY{p}{,}\PY{+w}{ }\PY{n}{ylabel}\PY{o}{=}\PY{l+s}{\PYZdq{}}\PY{l+s}{Observed mag}\PY{l+s}{\PYZdq{}}\PY{p}{)}
\PY{n}{errorbars!}\PY{p}{(}\PY{n}{data}\PY{o}{.}\PY{n}{z}\PY{p}{,}\PY{+w}{ }\PY{n}{data}\PY{o}{.}\PY{n}{mag}\PY{p}{,}\PY{+w}{ }\PY{n}{data}\PY{o}{.}\PY{n}{sigma\PYZus{}mag}\PY{p}{)}
\PY{n}{scatter!}\PY{p}{(}\PY{n}{data}\PY{o}{.}\PY{n}{z}\PY{p}{,}\PY{+w}{ }\PY{n}{data}\PY{o}{.}\PY{n}{mag}\PY{p}{,}\PY{+w}{ }\PY{n}{markersize}\PY{o}{=}\PY{l+m+mi}{5}\PY{p}{,}\PY{+w}{ }\PY{n}{color}\PY{o}{=}\PY{l+s+ss}{:maroon}\PY{p}{)}

\PY{c}{\PYZsh{} Compute the average absolute magnitude M given nominal cosmology \PYZhy{}\PYZhy{} ie, an estimate of the absolute mag of the supernovae}
\PY{n}{DLx}\PY{+w}{ }\PY{o}{=}\PY{+w}{ }\PY{n}{map}\PY{p}{(}\PY{n}{z}\PY{o}{\PYZhy{}\PYZgt{}}\PY{n}{distance\PYZus{}modulus}\PY{p}{(}\PY{n}{nominal}\PY{p}{,}\PY{+w}{ }\PY{n}{z}\PY{p}{)}\PY{p}{,}\PY{+w}{ }\PY{n}{data}\PY{o}{.}\PY{n}{z}\PY{p}{)}
\PY{n}{abs\PYZus{}mag}\PY{+w}{ }\PY{o}{=}\PY{+w}{ }\PY{n}{median}\PY{p}{(}\PY{n}{data}\PY{o}{.}\PY{n}{mag}\PY{+w}{ }\PY{o}{\PYZhy{}}\PY{+w}{ }\PY{n}{DLx}\PY{p}{)}

\PY{c}{\PYZsh{} Here\PYZsq{}s another way to plot a function evaluated on a grid of values.}
\PY{n}{zgrid}\PY{+w}{ }\PY{o}{=}\PY{+w}{ }\PY{l+m+mf}{0.01}\PY{o}{:}\PY{l+m+mf}{0.01}\PY{o}{:}\PY{l+m+mf}{1.}
\PY{n}{DL}\PY{+w}{ }\PY{o}{=}\PY{+w}{ }\PY{n}{map}\PY{p}{(}\PY{n}{z}\PY{o}{\PYZhy{}\PYZgt{}}\PY{n}{distance\PYZus{}modulus}\PY{p}{(}\PY{n}{nominal}\PY{p}{,}\PY{+w}{ }\PY{n}{z}\PY{p}{)}\PY{p}{,}\PY{+w}{ }\PY{n}{zgrid}\PY{p}{)}
\PY{n}{lines!}\PY{p}{(}\PY{n}{zgrid}\PY{p}{,}\PY{+w}{ }\PY{n}{DL}\PY{+w}{ }\PY{o}{.+}\PY{+w}{ }\PY{n}{abs\PYZus{}mag}\PY{p}{,}\PY{+w}{ }\PY{n}{label}\PY{o}{=}\PY{l+s}{\PYZdq{}}\PY{l+s}{Nominal OmegaM = 0.29, OmegaDE = 0.71}\PY{l+s}{\PYZdq{}}\PY{p}{)}

\PY{n}{universe}\PY{+w}{ }\PY{o}{=}\PY{+w}{ }\PY{n}{cosmology}\PY{p}{(}\PY{n}{OmegaK}\PY{o}{=}\PY{l+m+mf}{0.0}\PY{p}{,}\PY{+w}{ }\PY{n}{OmegaM}\PY{o}{=}\PY{l+m+mf}{0.6}\PY{p}{,}\PY{+w}{ }\PY{n}{Tcmb}\PY{o}{=}\PY{l+m+mi}{0}\PY{p}{)}
\PY{n}{DL}\PY{+w}{ }\PY{o}{=}\PY{+w}{ }\PY{n}{map}\PY{p}{(}\PY{n}{z}\PY{o}{\PYZhy{}\PYZgt{}}\PY{n}{distance\PYZus{}modulus}\PY{p}{(}\PY{n}{universe}\PY{p}{,}\PY{+w}{ }\PY{n}{z}\PY{p}{)}\PY{p}{,}\PY{+w}{ }\PY{n}{zgrid}\PY{p}{)}
\PY{n}{lines!}\PY{p}{(}\PY{n}{zgrid}\PY{p}{,}\PY{+w}{ }\PY{n}{DL}\PY{+w}{ }\PY{o}{.+}\PY{+w}{ }\PY{n}{abs\PYZus{}mag}\PY{p}{,}\PY{+w}{ }\PY{n}{color}\PY{o}{=}\PY{l+s+ss}{:red}\PY{p}{,}\PY{+w}{ }\PY{n}{label}\PY{o}{=}\PY{l+s}{\PYZdq{}}\PY{l+s}{OmegaM = 0.6, OmegaDE = 0.4}\PY{l+s}{\PYZdq{}}\PY{p}{)}

\PY{n}{universe}\PY{+w}{ }\PY{o}{=}\PY{+w}{ }\PY{n}{cosmology}\PY{p}{(}\PY{n}{OmegaK}\PY{o}{=}\PY{l+m+mf}{0.0}\PY{p}{,}\PY{+w}{ }\PY{n}{OmegaM}\PY{o}{=}\PY{l+m+mf}{0.1}\PY{p}{,}\PY{+w}{ }\PY{n}{Tcmb}\PY{o}{=}\PY{l+m+mi}{0}\PY{p}{)}
\PY{n}{DL}\PY{+w}{ }\PY{o}{=}\PY{+w}{ }\PY{n}{map}\PY{p}{(}\PY{n}{z}\PY{o}{\PYZhy{}\PYZgt{}}\PY{n}{distance\PYZus{}modulus}\PY{p}{(}\PY{n}{universe}\PY{p}{,}\PY{+w}{ }\PY{n}{z}\PY{p}{)}\PY{p}{,}\PY{+w}{ }\PY{n}{zgrid}\PY{p}{)}
\PY{n}{lines!}\PY{p}{(}\PY{n}{zgrid}\PY{p}{,}\PY{+w}{ }\PY{n}{DL}\PY{+w}{ }\PY{o}{.+}\PY{+w}{ }\PY{n}{abs\PYZus{}mag}\PY{p}{,}\PY{+w}{ }\PY{n}{color}\PY{o}{=}\PY{l+s+ss}{:green}\PY{p}{,}\PY{+w}{ }\PY{n}{label}\PY{o}{=}\PY{l+s}{\PYZdq{}}\PY{l+s}{OmegaM = 0.1, OmegaDE = 0.9}\PY{l+s}{\PYZdq{}}\PY{p}{)}

\PY{c}{\PYZsh{}f[2,1] = Legend(f, ax, \PYZdq{}Cosmologies\PYZdq{}, framevisible = false)}
\PY{c}{\PYZsh{} Create a legend for our plot}
\PY{n}{axislegend}\PY{p}{(}\PY{n}{ax}\PY{p}{,}\PY{+w}{ }\PY{n}{position}\PY{+w}{ }\PY{o}{=}\PY{+w}{ }\PY{l+s+ss}{:rb}\PY{p}{)}
\PY{n}{save}\PY{p}{(}\PY{l+s}{\PYZdq{}}\PY{l+s}{PerlmutterLines.png}\PY{l+s}{\PYZdq{}}\PY{p}{)}
\PY{n}{f}
\end{Verbatim}


\end{tcolorbox}

    \begin{center}
      \includegraphics[]{/home/jgouws/PSICourses/numericalMethods/PSI-Numerical-Methods-2024-MCMC-Homework1/PerlmutterLines.png}
    \end{center}

    \begin{tcolorbox}[breakable, size=fbox, boxrule=1pt, pad at break*=1mm,colback=cellbackground, colframe=cellborder]
\prompt{In}{incolor}{9}{\boxspacing}
\begin{Verbatim}[commandchars=\\\{\}]
\PY{c}{\PYZsh{} Here\PYZsq{}s our scalar estimate of the absolute mag.}
\PY{n}{abs\PYZus{}mag}
\end{Verbatim}
\end{tcolorbox}

            \begin{tcolorbox}[breakable, size=fbox, boxrule=.5pt, pad at break*=1mm, opacityfill=0]
\prompt{Out}{outcolor}{9}{\boxspacing}
\begin{Verbatim}[commandchars=\\\{\}]
-19.228824925301424
\end{Verbatim}
\end{tcolorbox}
        
    \hypertarget{part-1---the-log-likelihood-terrain}{%
\subsection{Part 1 - The Log-likelihood
terrain}\label{part-1---the-log-likelihood-terrain}}

    First, you have to write out the likelihood function for the observed
supernova data, given cosmological model parameters.

That is, please complete the following function. It will be passed
vectors of \texttt{z}, \texttt{mag}, and \texttt{mag\_error}
measurements, plus scalar parameters \texttt{M}, \texttt{Omega\_M} and
\texttt{Omega\_DE}. You will need to create a ``cosmology'' object, find
the \emph{distance modulus} for each redshift \texttt{z}, and add that
to the absolute mag \texttt{M} to get the \emph{predicted} magnitude.
You will then compare that to each measured magnitude, and compute the
likelihood.

    \begin{tcolorbox}[breakable, size=fbox, boxrule=1pt, pad at break*=1mm,colback=cellbackground, colframe=cellborder]
\prompt{In}{incolor}{10}{\boxspacing}
\begin{Verbatim}[commandchars=\\\{\}]
\PY{k}{function}\PY{+w}{ }\PY{n}{supernova\PYZus{}log\PYZus{}likelihood}\PY{p}{(}\PY{n}{zs}\PY{p}{,}\PY{+w}{ }\PY{n}{mag}\PY{p}{,}\PY{+w}{ }\PY{n}{mag\PYZus{}error}\PY{p}{,}\PY{+w}{ }\PY{n}{M}\PY{p}{,}\PY{+w}{ }\PY{n}{Omatter}\PY{p}{,}\PY{+w}{ }\PY{n}{Ode}\PY{p}{)}
\PY{+w}{    }\PY{c}{\PYZsh{} z: vector of redshifts}
\PY{+w}{    }\PY{c}{\PYZsh{} mag: vector of measured magnitudes}
\PY{+w}{    }\PY{c}{\PYZsh{} mag\PYZus{}error: vector of uncertainties on the measured magnitudes (sigmas).}
\PY{+w}{    }\PY{c}{\PYZsh{} M: scalar, absolute magnitude of a Type\PYZhy{}1a supernova}
\PY{+w}{    }\PY{c}{\PYZsh{} Omatter: scalar Omega\PYZus{}M, amount of matter in the universe}
\PY{+w}{    }\PY{c}{\PYZsh{} Ode: scalar Omega\PYZus{}DE, amount of dark energy in the universe}

\PY{+w}{    }\PY{c}{\PYZsh{}\PYZsh{}\PYZsh{}   YOUR CODE HERE!!}
\PY{+w}{    }\PY{n}{cosModel}\PY{+w}{ }\PY{o}{=}\PY{+w}{ }\PY{n}{cosmology}\PY{p}{(}\PY{n}{OmegaK}\PY{o}{=}\PY{l+m+mf}{1.}\PY{+w}{ }\PY{o}{\PYZhy{}}\PY{+w}{ }\PY{n}{Omatter}\PY{+w}{ }\PY{o}{\PYZhy{}}\PY{+w}{ }\PY{n}{Ode}\PY{p}{,}\PY{+w}{ }\PY{n}{OmegaM}\PY{o}{=}\PY{n}{Omatter}\PY{p}{,}\PY{+w}{ }\PY{n}{Tcmb}\PY{o}{=}\PY{l+m+mi}{0}\PY{p}{)}\PY{+w}{ }\PY{c}{\PYZsh{}make the cosmology \PYZdq{}object\PYZdq{}}

\PY{+w}{    }\PY{n}{mag\PYZus{}pred}\PY{+w}{ }\PY{o}{=}\PY{+w}{ }\PY{p}{(}\PY{n}{z}\PY{+w}{ }\PY{o}{\PYZhy{}\PYZgt{}}\PY{+w}{ }\PY{n}{distance\PYZus{}modulus}\PY{p}{(}\PY{n}{cosModel}\PY{p}{,}\PY{+w}{ }\PY{n}{z}\PY{p}{)}\PY{+w}{ }\PY{o}{+}\PY{+w}{ }\PY{n}{M}\PY{p}{)}\PY{o}{.}\PY{p}{(}\PY{n}{zs}\PY{p}{)}\PY{+w}{ }\PY{c}{\PYZsh{}vectorize the operations on zs}

\PY{+w}{    }\PY{n}{chi}\PY{+w}{ }\PY{o}{=}\PY{+w}{ }\PY{p}{(}\PY{n}{mag\PYZus{}pred}\PY{+w}{ }\PY{o}{\PYZhy{}}\PY{+w}{ }\PY{n}{mag}\PY{p}{)}\PY{+w}{ }\PY{o}{./}\PY{+w}{ }\PY{n}{mag\PYZus{}error}
\PY{+w}{    }\PY{n}{sum}\PY{p}{(}\PY{o}{\PYZhy{}}\PY{l+m+mf}{0.5}\PY{+w}{ }\PY{o}{.*}\PY{+w}{ }\PY{n}{chi}\PY{o}{.\PYZca{}}\PY{l+m+mi}{2}\PY{p}{)}

\PY{+w}{    }\PY{c}{\PYZsh{} You must return a scalar value}
\PY{k}{end}\PY{p}{;}
\end{Verbatim}
\end{tcolorbox}

    Next, please keep \texttt{M} fixed to the \texttt{abs\_mag} value we
computed above, and call your \texttt{supernova\_log\_likelihood} on a
grid of \texttt{Omega\_M} and \texttt{Omega\_DE} values. (You will pass
in \texttt{data.z}, \texttt{data.mag}, and \texttt{data.sigma\_mag} for
the \texttt{z}, \texttt{mag}, and \texttt{mag\_error} values.)

Try a grid from 0 to 1 for both Omega\_M and Omega\_DE, and show the
\texttt{supernova\_log\_likelihood} values using the \texttt{heatmap}
function. You may find it helpful to limit the range using something
like
\texttt{heatmap(om\_grid,\ ode\_grid,\ sn\_ll,\ colorrange={[}maximum(sn\_ll)-20,\ maximum(sn\_ll){]})}.

Another thing you can do is, instead of showing the
\emph{log}-likelihood, show the likelihood by taking the \texttt{exp} of
your \texttt{sn\_ll} grid, like this,
\texttt{heatmap(om\_grid,\ ode\_grid,\ exp.(sn\_ll))}.

Please compare your plot to Figure 7 in the Perlmutter et al.~1999
paper, shown below. Does your likelihood contour look consistent with
the blue ellipses?

    \begin{tcolorbox}[breakable, size=fbox, boxrule=1pt, pad at break*=1mm,colback=cellbackground, colframe=cellborder]
\prompt{In}{incolor}{28}{\boxspacing}
\begin{Verbatim}[commandchars=\\\{\}]
\PY{n}{n\PYZus{}om}\PY{p}{,}\PY{+w}{ }\PY{n}{n\PYZus{}ode}\PY{+w}{ }\PY{o}{=}\PY{+w}{ }\PY{l+m+mi}{50}\PY{p}{,}\PY{l+m+mi}{50}
\PY{n}{om\PYZus{}grid}\PY{+w}{ }\PY{o}{=}\PY{+w}{ }\PY{k+kt}{LinRange}\PY{p}{(}\PY{l+m+mi}{0}\PY{p}{,}\PY{+w}{ }\PY{l+m+mi}{1}\PY{p}{,}\PY{+w}{ }\PY{n}{n\PYZus{}om}\PY{p}{)}
\PY{n}{ode\PYZus{}grid}\PY{+w}{ }\PY{o}{=}\PY{+w}{ }\PY{k+kt}{LinRange}\PY{p}{(}\PY{l+m+mi}{0}\PY{p}{,}\PY{+w}{ }\PY{l+m+mi}{1}\PY{p}{,}\PY{+w}{ }\PY{n}{n\PYZus{}ode}\PY{p}{)}
\PY{n}{sn\PYZus{}ll}\PY{+w}{ }\PY{o}{=}\PY{+w}{ }\PY{n}{zeros}\PY{p}{(}\PY{n}{n\PYZus{}om}\PY{p}{,}\PY{+w}{ }\PY{n}{n\PYZus{}ode}\PY{p}{)}
\PY{k}{for}\PY{+w}{ }\PY{n}{i}\PY{+w}{ }\PY{k}{in}\PY{+w}{ }\PY{l+m+mi}{1}\PY{o}{:}\PY{n}{n\PYZus{}om}
\PY{+w}{    }\PY{k}{for}\PY{+w}{ }\PY{n}{j}\PY{+w}{ }\PY{k}{in}\PY{+w}{ }\PY{l+m+mi}{1}\PY{o}{:}\PY{n}{n\PYZus{}ode}
\PY{+w}{        }\PY{n}{sn\PYZus{}ll}\PY{p}{[}\PY{n}{i}\PY{p}{,}\PY{+w}{ }\PY{n}{j}\PY{p}{]}\PY{+w}{ }\PY{o}{=}\PY{+w}{ }\PY{n}{supernova\PYZus{}log\PYZus{}likelihood}\PY{p}{(}\PY{n}{data}\PY{o}{.}\PY{n}{z}\PY{p}{,}\PY{+w}{ }\PY{n}{data}\PY{o}{.}\PY{n}{mag}\PY{p}{,}\PY{+w}{ }\PY{n}{data}\PY{o}{.}\PY{n}{sigma\PYZus{}mag}\PY{p}{,}\PY{+w}{ }\PY{n}{abs\PYZus{}mag}\PY{p}{,}\PY{+w}{ }\PY{n}{om\PYZus{}grid}\PY{p}{[}\PY{n}{i}\PY{p}{]}\PY{p}{,}\PY{+w}{ }\PY{n}{ode\PYZus{}grid}\PY{p}{[}\PY{n}{j}\PY{p}{]}\PY{p}{)}
\PY{+w}{    }\PY{k}{end}
\PY{k}{end}
\PY{n}{f}\PY{+w}{ }\PY{o}{=}\PY{+w}{ }\PY{n}{Figure}\PY{p}{(}\PY{p}{)}
\PY{n}{ax}\PY{+w}{ }\PY{o}{=}\PY{+w}{ }\PY{n}{Axis}\PY{p}{(}\PY{n}{f}\PY{p}{[}\PY{l+m+mi}{1}\PY{p}{,}\PY{+w}{ }\PY{l+m+mi}{1}\PY{p}{]}\PY{p}{)}

\PY{n}{heatmap!}\PY{p}{(}\PY{n}{om\PYZus{}grid}\PY{p}{,}\PY{+w}{ }\PY{n}{ode\PYZus{}grid}\PY{p}{,}\PY{+w}{ }\PY{n}{sn\PYZus{}ll}\PY{p}{,}\PY{+w}{ }\PY{n}{colorrange}\PY{o}{=}\PY{p}{[}\PY{n}{maximum}\PY{p}{(}\PY{n}{sn\PYZus{}ll}\PY{p}{)}\PY{o}{\PYZhy{}}\PY{l+m+mi}{20}\PY{p}{,}\PY{+w}{ }\PY{n}{maximum}\PY{p}{(}\PY{n}{sn\PYZus{}ll}\PY{p}{)}\PY{p}{]}\PY{p}{)}

\PY{n}{save}\PY{p}{(}\PY{l+s}{\PYZdq{}}\PY{l+s}{FirstHeatmapLimitedRange.png}\PY{l+s}{\PYZdq{}}\PY{p}{,}\PY{+w}{ }\PY{n}{f}\PY{p}{)}
\PY{n}{f}
\end{Verbatim}
\end{tcolorbox}

    \begin{center}
      \includegraphics[]{FirstHeatmapLimitedRange.png}
    \end{center}

    
    \begin{Verbatim}[commandchars=\\\{\}]
Bonito.App(Bonito.var"\#8\#14"\{WGLMakie.var"\#20\#21"\{WGLMakie.Screen, Scene\}\}(WGLMakie.var"\#20\#21"\{WGLMakie.Screen, Scene\}(WGLMakie.Screen(
    framerate = 30.0,
    resize\_to = nothing,
    px\_per\_unit = automatic,
    scalefactor = automatic,
    session = nothing,
    three = Channel\{Bool\}(1) (empty),
    scene = Scene (600px, 450px):
  0 Plots
  1 Child Scene:
    └ Scene (600px, 450px),
), Scene (600px, 450px):
  0 Plots
  1 Child Scene:
    └ Scene (600px, 450px))), Base.RefValue\{Union\{Nothing, Bonito.Session\}\}(nothing), "Bonito App", false)
    \end{Verbatim}

    
    The resulting heat map looks pretty similar to the contour plot in
Perlmutter, same shape and direction in the restricted region.

    

    Next, try expanding the grid ranges for Omega\_M and Omega\_DE up to,
say, 0 to 2 or 0 to 3. You should encounter a problem -- the cosmology
package will fail to compute the \texttt{distance\_modulus} for some
combinations! You can work around this by using Julia's
\texttt{try...catch} syntax, like this:

    \begin{tcolorbox}[breakable, size=fbox, boxrule=1pt, pad at break*=1mm,colback=cellbackground, colframe=cellborder]
\prompt{In}{incolor}{12}{\boxspacing}
\begin{Verbatim}[commandchars=\\\{\}]
\PY{n}{n\PYZus{}om2}\PY{p}{,}\PY{+w}{ }\PY{n}{n\PYZus{}ode2}\PY{+w}{ }\PY{o}{=}\PY{+w}{ }\PY{l+m+mi}{100}\PY{p}{,}\PY{l+m+mi}{100}
\PY{n}{om\PYZus{}grid2}\PY{+w}{ }\PY{o}{=}\PY{+w}{ }\PY{k+kt}{LinRange}\PY{p}{(}\PY{l+m+mi}{0}\PY{p}{,}\PY{+w}{ }\PY{l+m+mi}{3}\PY{p}{,}\PY{+w}{ }\PY{n}{n\PYZus{}om2}\PY{p}{)}
\PY{n}{ode\PYZus{}grid2}\PY{+w}{ }\PY{o}{=}\PY{+w}{ }\PY{k+kt}{LinRange}\PY{p}{(}\PY{o}{\PYZhy{}}\PY{l+m+mf}{1.5}\PY{p}{,}\PY{+w}{ }\PY{l+m+mi}{3}\PY{p}{,}\PY{+w}{ }\PY{n}{n\PYZus{}ode2}\PY{p}{)}
\PY{n}{sn\PYZus{}ll2}\PY{+w}{ }\PY{o}{=}\PY{+w}{ }\PY{n}{zeros}\PY{p}{(}\PY{n}{n\PYZus{}om2}\PY{p}{,}\PY{+w}{ }\PY{n}{n\PYZus{}ode2}\PY{p}{)}
\PY{k}{for}\PY{+w}{ }\PY{n}{i}\PY{+w}{ }\PY{k}{in}\PY{+w}{ }\PY{l+m+mi}{1}\PY{o}{:}\PY{n}{n\PYZus{}om2}
\PY{+w}{    }\PY{k}{for}\PY{+w}{ }\PY{n}{j}\PY{+w}{ }\PY{k}{in}\PY{+w}{ }\PY{l+m+mi}{1}\PY{o}{:}\PY{n}{n\PYZus{}ode2}
\PY{+w}{        }\PY{k}{try}\PY{+w}{ }\PY{c}{\PYZsh{}very bad way of doing this, should rather make supernova\PYZus{}log\PYZus{}likelihood to check for parameters.}
\PY{+w}{            }\PY{n}{sn\PYZus{}ll2}\PY{p}{[}\PY{n}{i}\PY{p}{,}\PY{+w}{ }\PY{n}{j}\PY{p}{]}\PY{+w}{ }\PY{o}{=}\PY{+w}{ }\PY{n}{supernova\PYZus{}log\PYZus{}likelihood}\PY{p}{(}\PY{n}{data}\PY{o}{.}\PY{n}{z}\PY{p}{,}\PY{+w}{ }\PY{n}{data}\PY{o}{.}\PY{n}{mag}\PY{p}{,}\PY{+w}{ }\PY{n}{data}\PY{o}{.}\PY{n}{sigma\PYZus{}mag}\PY{p}{,}\PY{+w}{ }\PY{n}{abs\PYZus{}mag}\PY{p}{,}\PY{+w}{ }\PY{n}{om\PYZus{}grid2}\PY{p}{[}\PY{n}{i}\PY{p}{]}\PY{p}{,}\PY{+w}{ }\PY{n}{ode\PYZus{}grid2}\PY{p}{[}\PY{n}{j}\PY{p}{]}\PY{p}{)}
\PY{+w}{        }\PY{k}{catch}\PY{+w}{ }\PY{n}{err}
\PY{+w}{            }\PY{n}{sn\PYZus{}ll2}\PY{p}{[}\PY{n}{i}\PY{p}{,}\PY{+w}{ }\PY{n}{j}\PY{p}{]}\PY{+w}{ }\PY{o}{=}\PY{+w}{ }\PY{o}{\PYZhy{}}\PY{n+nb}{Inf}
\PY{+w}{        }\PY{k}{end}
\PY{+w}{    }\PY{k}{end}
\PY{k}{end}

\PY{n}{f}\PY{+w}{ }\PY{o}{=}\PY{+w}{ }\PY{n}{Figure}\PY{p}{(}\PY{p}{)}
\PY{n}{ax}\PY{+w}{ }\PY{o}{=}\PY{+w}{ }\PY{n}{Axis}\PY{p}{(}\PY{n}{f}\PY{p}{[}\PY{l+m+mi}{1}\PY{p}{,}\PY{+w}{ }\PY{l+m+mi}{1}\PY{p}{]}\PY{p}{)}

\PY{n}{heatmap!}\PY{p}{(}\PY{n}{om\PYZus{}grid2}\PY{p}{,}\PY{+w}{ }\PY{n}{ode\PYZus{}grid2}\PY{p}{,}\PY{+w}{ }\PY{n}{sn\PYZus{}ll2}\PY{p}{,}\PY{+w}{ }\PY{n}{colorrange}\PY{o}{=}\PY{p}{[}\PY{n}{maximum}\PY{p}{(}\PY{n}{sn\PYZus{}ll2}\PY{p}{)}\PY{o}{\PYZhy{}}\PY{l+m+mi}{20}\PY{p}{,}\PY{+w}{ }\PY{n}{maximum}\PY{p}{(}\PY{n}{sn\PYZus{}ll2}\PY{p}{)}\PY{p}{]}\PY{p}{)}

\PY{n}{save}\PY{p}{(}\PY{l+s}{\PYZdq{}}\PY{l+s}{HeatmapExtendedRange.png}\PY{l+s}{\PYZdq{}}\PY{p}{,}\PY{+w}{ }\PY{n}{f}\PY{p}{)}
\PY{n}{f}
\end{Verbatim}
\end{tcolorbox}

    \begin{center}
      
      \includegraphics[]{HeatmapExtendedRange.png}
    \end{center}

    
    \begin{Verbatim}[commandchars=\\\{\}]
Bonito.App(Bonito.var"\#8\#14"\{WGLMakie.var"\#20\#21"\{WGLMakie.Screen, Scene\}\}(WGLMakie.var"\#20\#21"\{WGLMakie.Screen, Scene\}(WGLMakie.Screen(
    framerate = 30.0,
    resize\_to = nothing,
    px\_per\_unit = automatic,
    scalefactor = automatic,
    session = nothing,
    three = Channel\{Bool\}(1) (empty),
    scene = Scene (600px, 450px):
  0 Plots
  1 Child Scene:
    └ Scene (600px, 450px),
), Scene (600px, 450px):
  0 Plots
  1 Child Scene:
    └ Scene (600px, 450px))), Base.RefValue\{Union\{Nothing, Bonito.Session\}\}(nothing), "Bonito App", false)
    \end{Verbatim}

    
    This will ``try'' to run the \texttt{supernova\_log\_likelihood}
function, and if it fails, it will go into the ``catch'' branch.

    \hypertarget{part-2---using-mcmc-to-sample-from-the-likelihood}{%
\subsection{Part 2 - Using MCMC to sample from the
likelihood}\label{part-2---using-mcmc-to-sample-from-the-likelihood}}

    Next, we will use Markov Chain Monte Carlo to draw samples from the
likelihood distribution.

You can start with the \texttt{mcmc} function from the lecture.

You will need to tune the MCMC proposal's step sizes (also known as
``jump sizes''). To do this, you can use the variant of the
\texttt{mcmc} routine that cycles through the parameters and only jumps
one at a time, named \texttt{mcmc\_cyclic} in the updated lecture
notebook. After tuning the step sizes with \texttt{mcmc\_cyclic}, you
can go back to the plain \texttt{mcmc} routine if you want, or stick
with \texttt{mcmc\_cyclic}; it is up to you.

Please plot the samples from your MCMC chains, to demonstrate that the
chain looks like it has converged. Ideally, you would like to see
reasonable acceptance rates, and you would like to see the samples
``exploring'' the parameter space. Decide how many step you need to run
the MCMC routine for, and write a sentence or two describing why you
think that's a good number.

For this part, please include the \texttt{M} (absolute magnitude) as a
parameter that you are fitting -- so you are fitting for \texttt{M} in
addition to \texttt{Omega\_M} and \texttt{Omega\_DE}. This is a quite
standard situation where you have a ``nuisance'' parameter \texttt{M}
that you don't really care about, in addition to the \texttt{Omega}
parameters that you do care about.

    It is quite common to plot the results from an MCMC sampling using a
``corner plot'', which shows the distribution of each of the individual
parameters, and the joint distributions of pairs of parameters. This
will help you determine whether some of the parameters are correlated
with each other.

Below is a function you can use to generate corner plots from your chain
-- call it like
\texttt{cornerplot(chain,\ {[}"M",\ "Omega\_M",\ "Omega\_DE"{]})}. There
is also a CornerPlot package (https://juliapackages.com/p/cornerplot)
but I have not had luck getting it to work for me.

Once you have made you corner plots, please write a few sentences
interpreting what you see. Is the nuisance parameter \texttt{M}
correlated with the Omegas? Are the Omegas correlated with each other?

    \begin{tcolorbox}[breakable, size=fbox, boxrule=1pt, pad at break*=1mm,colback=cellbackground, colframe=cellborder]
\prompt{In}{incolor}{13}{\boxspacing}
\begin{Verbatim}[commandchars=\\\{\}]
\PY{k}{function}\PY{+w}{ }\PY{n}{cornerplot}\PY{p}{(}\PY{n}{x}\PY{p}{,}\PY{+w}{ }\PY{n}{names}\PY{p}{;}\PY{+w}{ }\PY{n}{figsize}\PY{o}{=}\PY{p}{(}\PY{l+m+mi}{600}\PY{p}{,}\PY{l+m+mi}{600}\PY{p}{)}\PY{p}{)}
\PY{+w}{    }\PY{c}{\PYZsh{} how many columns of data}
\PY{+w}{    }\PY{n}{dim}\PY{+w}{ }\PY{o}{=}\PY{+w}{ }\PY{n}{size}\PY{p}{(}\PY{n}{x}\PY{p}{,}\PY{+w}{ }\PY{l+m+mi}{2}\PY{p}{)}
\PY{+w}{    }\PY{c}{\PYZsh{} rows to plot}
\PY{+w}{    }\PY{n}{idxs}\PY{+w}{ }\PY{o}{=}\PY{+w}{ }\PY{l+m+mi}{1}\PY{o}{:}\PY{n}{size}\PY{p}{(}\PY{n}{x}\PY{p}{,}\PY{l+m+mi}{1}\PY{p}{)}
\PY{+w}{    }\PY{n}{f}\PY{+w}{ }\PY{o}{=}\PY{+w}{ }\PY{n}{Figure}\PY{p}{(}\PY{n}{size}\PY{o}{=}\PY{n}{figsize}\PY{p}{)}
\PY{+w}{    }\PY{k}{for}\PY{+w}{ }\PY{n}{i}\PY{+w}{ }\PY{k}{in}\PY{+w}{ }\PY{l+m+mi}{1}\PY{o}{:}\PY{n}{dim}\PY{p}{,}\PY{+w}{ }\PY{n}{j}\PY{+w}{ }\PY{k}{in}\PY{+w}{ }\PY{l+m+mi}{1}\PY{o}{:}\PY{n}{dim}
\PY{+w}{        }\PY{k}{if}\PY{+w}{ }\PY{n}{i}\PY{+w}{ }\PY{o}{\PYZlt{}}\PY{+w}{ }\PY{n}{j}
\PY{+w}{            }\PY{k}{continue}
\PY{+w}{        }\PY{k}{end}
\PY{+w}{        }\PY{n}{ax}\PY{+w}{ }\PY{o}{=}\PY{+w}{ }\PY{n}{Axis}\PY{p}{(}\PY{n}{f}\PY{p}{[}\PY{n}{i}\PY{p}{,}\PY{+w}{ }\PY{n}{j}\PY{p}{]}\PY{p}{,}\PY{+w}{ }\PY{n}{aspect}\PY{+w}{ }\PY{o}{=}\PY{+w}{ }\PY{l+m+mi}{1}\PY{p}{,}
\PY{+w}{                  }\PY{n}{topspinevisible}\PY{+w}{ }\PY{o}{=}\PY{+w}{ }\PY{n+nb}{false}\PY{p}{,}
\PY{+w}{                  }\PY{n}{rightspinevisible}\PY{+w}{ }\PY{o}{=}\PY{+w}{ }\PY{n+nb}{false}\PY{p}{,}\PY{p}{)}
\PY{+w}{        }\PY{k}{if}\PY{+w}{ }\PY{n}{i}\PY{+w}{ }\PY{o}{==}\PY{+w}{ }\PY{n}{j}
\PY{+w}{            }\PY{n}{hist!}\PY{p}{(}\PY{n}{x}\PY{p}{[}\PY{n}{idxs}\PY{p}{,}\PY{n}{i}\PY{p}{]}\PY{p}{,}\PY{+w}{ }\PY{n}{direction}\PY{o}{=}\PY{l+s+ss}{:y}\PY{p}{)}
\PY{+w}{            }\PY{n}{ax}\PY{o}{.}\PY{n}{xlabel}\PY{+w}{ }\PY{o}{=}\PY{+w}{ }\PY{n}{names}\PY{p}{[}\PY{n}{i}\PY{p}{]}
\PY{+w}{        }\PY{k}{else}
\PY{+w}{            }\PY{c}{\PYZsh{}scatter!(x[idxs,j], x[idxs,i], markersize=4)}
\PY{+w}{            }\PY{n}{hexbin!}\PY{p}{(}\PY{n}{x}\PY{p}{[}\PY{n}{idxs}\PY{p}{,}\PY{n}{j}\PY{p}{]}\PY{p}{,}\PY{+w}{ }\PY{n}{x}\PY{p}{[}\PY{n}{idxs}\PY{p}{,}\PY{n}{i}\PY{p}{]}\PY{p}{)}
\PY{+w}{            }\PY{n}{ax}\PY{o}{.}\PY{n}{xlabel}\PY{+w}{ }\PY{o}{=}\PY{+w}{ }\PY{n}{names}\PY{p}{[}\PY{n}{j}\PY{p}{]}
\PY{+w}{            }\PY{n}{ax}\PY{o}{.}\PY{n}{ylabel}\PY{+w}{ }\PY{o}{=}\PY{+w}{ }\PY{n}{names}\PY{p}{[}\PY{n}{i}\PY{p}{]}
\PY{+w}{        }\PY{k}{end}
\PY{+w}{    }\PY{k}{end}
\PY{+w}{    }\PY{n}{f}
\PY{k}{end}\PY{p}{;}
\end{Verbatim}
\end{tcolorbox}

    Here is the cyclic Markov Chain Monte Carlo function that I got from the
lecture notes.

    \begin{tcolorbox}[breakable, size=fbox, boxrule=1pt, pad at break*=1mm,colback=cellbackground, colframe=cellborder]
\prompt{In}{incolor}{14}{\boxspacing}
\begin{Verbatim}[commandchars=\\\{\}]
\PY{k}{function}\PY{+w}{ }\PY{n}{mcmc\PYZus{}cyclic}\PY{p}{(}\PY{n}{logprob\PYZus{}func}\PY{p}{,}\PY{+w}{ }\PY{n}{propose\PYZus{}func}\PY{p}{,}\PY{+w}{ }\PY{n}{initial\PYZus{}p}\PY{p}{,}\PY{+w}{ }\PY{n}{n\PYZus{}steps}\PY{p}{)}
\PY{+w}{    }\PY{n}{p}\PY{+w}{ }\PY{o}{=}\PY{+w}{ }\PY{n}{initial\PYZus{}p}
\PY{+w}{    }\PY{n}{logprob}\PY{+w}{ }\PY{o}{=}\PY{+w}{ }\PY{n}{logprob\PYZus{}func}\PY{p}{(}\PY{n}{p}\PY{p}{)}
\PY{+w}{    }\PY{n}{chain}\PY{+w}{ }\PY{o}{=}\PY{+w}{ }\PY{n}{zeros}\PY{p}{(}\PY{n}{n\PYZus{}steps}\PY{p}{,}\PY{+w}{ }\PY{n}{length}\PY{p}{(}\PY{n}{p}\PY{p}{)}\PY{p}{)}
\PY{+w}{    }\PY{n}{n\PYZus{}accept}\PY{+w}{ }\PY{o}{=}\PY{+w}{ }\PY{n}{zeros}\PY{p}{(}\PY{n}{length}\PY{p}{(}\PY{n}{p}\PY{p}{)}\PY{p}{)}

\PY{+w}{    }\PY{k}{for}\PY{+w}{ }\PY{n}{i}\PY{+w}{ }\PY{k}{in}\PY{+w}{ }\PY{l+m+mi}{1}\PY{o}{:}\PY{n}{n\PYZus{}steps}

\PY{+w}{        }\PY{c}{\PYZsh{} We\PYZsq{}re going to update one index at a time... 1, 2, 1, 2, ....}
\PY{+w}{        }\PY{n}{update\PYZus{}index}\PY{+w}{ }\PY{o}{=}\PY{+w}{ }\PY{l+m+mi}{1}\PY{+w}{ }\PY{o}{+}\PY{+w}{ }\PY{p}{(}\PY{p}{(}\PY{n}{i}\PY{o}{\PYZhy{}}\PY{l+m+mi}{1}\PY{p}{)}\PY{+w}{ }\PY{o}{\PYZpc{}}\PY{+w}{ }\PY{n}{length}\PY{p}{(}\PY{n}{p}\PY{p}{)}\PY{p}{)}

\PY{+w}{        }\PY{c}{\PYZsh{} Call the proposal function to generate new values for all parameters...}
\PY{+w}{        }\PY{n}{p\PYZus{}prop}\PY{+w}{ }\PY{o}{=}\PY{+w}{ }\PY{n}{propose\PYZus{}func}\PY{p}{(}\PY{n}{p}\PY{p}{)}
\PY{+w}{        }\PY{c}{\PYZsh{} ... but then only keep one of the new parameter values!}
\PY{+w}{        }\PY{n}{p\PYZus{}new}\PY{+w}{ }\PY{o}{=}\PY{+w}{ }\PY{n}{copy}\PY{p}{(}\PY{n}{p}\PY{p}{)}
\PY{+w}{        }\PY{n}{p\PYZus{}new}\PY{p}{[}\PY{n}{update\PYZus{}index}\PY{p}{]}\PY{+w}{ }\PY{o}{=}\PY{+w}{ }\PY{n}{p\PYZus{}prop}\PY{p}{[}\PY{n}{update\PYZus{}index}\PY{p}{]}
\PY{+w}{        }
\PY{+w}{        }\PY{n}{logprob\PYZus{}new}\PY{+w}{ }\PY{o}{=}\PY{+w}{ }\PY{n}{logprob\PYZus{}func}\PY{p}{(}\PY{n}{p\PYZus{}new}\PY{p}{)}

\PY{+w}{        }\PY{n}{ratio}\PY{+w}{ }\PY{o}{=}\PY{+w}{ }\PY{n}{exp}\PY{p}{(}\PY{n}{logprob\PYZus{}new}\PY{+w}{ }\PY{o}{\PYZhy{}}\PY{+w}{ }\PY{n}{logprob}\PY{p}{)}
\PY{+w}{        }\PY{k}{if}\PY{+w}{ }\PY{n}{ratio}\PY{+w}{ }\PY{o}{\PYZgt{}}\PY{+w}{ }\PY{l+m+mi}{1}
\PY{+w}{            }\PY{c}{\PYZsh{} Jump to the new place}
\PY{+w}{            }\PY{n}{p}\PY{+w}{ }\PY{o}{=}\PY{+w}{ }\PY{n}{p\PYZus{}new}
\PY{+w}{            }\PY{n}{logprob}\PY{+w}{ }\PY{o}{=}\PY{+w}{ }\PY{n}{logprob\PYZus{}new}
\PY{+w}{            }\PY{n}{n\PYZus{}accept}\PY{p}{[}\PY{n}{update\PYZus{}index}\PY{p}{]}\PY{+w}{ }\PY{o}{+=}\PY{+w}{ }\PY{l+m+mi}{1}
\PY{+w}{        }\PY{k}{else}
\PY{+w}{            }\PY{c}{\PYZsh{} Jump to the new place with probability \PYZdq{}ratio\PYZdq{}}
\PY{+w}{            }\PY{n}{u}\PY{+w}{ }\PY{o}{=}\PY{+w}{ }\PY{n}{rand}\PY{p}{(}\PY{p}{)}
\PY{+w}{            }\PY{k}{if}\PY{+w}{ }\PY{n}{u}\PY{+w}{ }\PY{o}{\PYZlt{}}\PY{+w}{ }\PY{n}{ratio}
\PY{+w}{                }\PY{c}{\PYZsh{} Jump to the new place}
\PY{+w}{                }\PY{n}{p}\PY{+w}{ }\PY{o}{=}\PY{+w}{ }\PY{n}{p\PYZus{}new}
\PY{+w}{                }\PY{n}{logprob}\PY{+w}{ }\PY{o}{=}\PY{+w}{ }\PY{n}{logprob\PYZus{}new}
\PY{+w}{                }\PY{n}{n\PYZus{}accept}\PY{p}{[}\PY{n}{update\PYZus{}index}\PY{p}{]}\PY{+w}{ }\PY{o}{+=}\PY{+w}{ }\PY{l+m+mi}{1}
\PY{+w}{            }\PY{k}{else}
\PY{+w}{                }\PY{c}{\PYZsh{} Stay where we are}
\PY{+w}{            }\PY{k}{end}
\PY{+w}{        }\PY{k}{end}
\PY{+w}{        }\PY{n}{chain}\PY{p}{[}\PY{n}{i}\PY{p}{,}\PY{+w}{ }\PY{l+m+mi}{1}\PY{o}{:}\PY{k}{end}\PY{p}{]}\PY{+w}{ }\PY{o}{=}\PY{+w}{ }\PY{n}{p}
\PY{+w}{    }\PY{k}{end}
\PY{+w}{    }\PY{c}{\PYZsh{} The number of times we step each parameter is roughly n\PYZus{}steps / n\PYZus{}parameters}
\PY{+w}{    }\PY{n}{chain}\PY{p}{,}\PY{+w}{ }\PY{n}{n\PYZus{}accept}\PY{+w}{ }\PY{o}{./}\PY{+w}{ }\PY{p}{(}\PY{n}{n\PYZus{}steps}\PY{+w}{ }\PY{o}{./}\PY{+w}{ }\PY{n}{length}\PY{p}{(}\PY{n}{p}\PY{p}{)}\PY{p}{)}
\PY{k}{end}\PY{p}{;}
\end{Verbatim}
\end{tcolorbox}

    Here I run the cyclic Monte Carlo Markov chain so that I can tune the
parameters. I went for more samples and smaller jumps as this function
does not take too long to execute. I took a look at acceptance rates too
to make sure that most jumps were good jumps, I chose the smaller jumps
size to increase the acceptance rate of sampling with all paramters
having similar acceptance rates.

    \begin{tcolorbox}[breakable, size=fbox, boxrule=1pt, pad at break*=1mm,colback=cellbackground, colframe=cellborder]
\prompt{In}{incolor}{15}{\boxspacing}
\begin{Verbatim}[commandchars=\\\{\}]
\PY{k}{function}\PY{+w}{ }\PY{n}{propose}\PY{p}{(}\PY{n}{p}\PY{p}{,}\PY{+w}{ }\PY{n}{jump\PYZus{}sizes}\PY{p}{)}
\PY{+w}{    }\PY{n}{p}\PY{+w}{ }\PY{o}{.+}\PY{+w}{ }\PY{n}{randn}\PY{p}{(}\PY{n}{length}\PY{p}{(}\PY{n}{p}\PY{p}{)}\PY{p}{)}\PY{+w}{ }\PY{o}{.*}\PY{+w}{ }\PY{n}{jump\PYZus{}sizes}
\PY{k}{end}\PY{p}{;}

\PY{n}{initial\PYZus{}guess}\PY{+w}{ }\PY{o}{=}\PY{+w}{ }\PY{p}{[}\PY{n}{abs\PYZus{}mag}\PY{p}{,}\PY{+w}{ }\PY{l+m+mf}{.5}\PY{p}{,}\PY{+w}{ }\PY{l+m+mf}{.5}\PY{p}{]}
\PY{n}{jump\PYZus{}sizes}\PY{+w}{ }\PY{o}{=}\PY{+w}{ }\PY{p}{[}\PY{l+m+mf}{0.02}\PY{p}{,}\PY{+w}{ }\PY{l+m+mf}{0.09}\PY{p}{,}\PY{+w}{ }\PY{l+m+mf}{0.1}\PY{p}{]}

\PY{n}{chain}\PY{p}{,}\PY{+w}{ }\PY{n}{accept\PYZus{}rate}\PY{+w}{ }\PY{o}{=}\PY{+w}{ }\PY{n}{mcmc\PYZus{}cyclic}\PY{p}{(}\PY{n}{p}\PY{+w}{ }\PY{o}{\PYZhy{}\PYZgt{}}\PY{+w}{ }
\PY{+w}{        }\PY{n}{supernova\PYZus{}log\PYZus{}likelihood}\PY{p}{(}\PY{n}{data}\PY{o}{.}\PY{n}{z}\PY{p}{,}\PY{+w}{ }\PY{n}{data}\PY{o}{.}\PY{n}{mag}\PY{p}{,}\PY{+w}{ }\PY{n}{data}\PY{o}{.}\PY{n}{sigma\PYZus{}mag}\PY{p}{,}\PY{+w}{ }\PY{n}{p}\PY{p}{[}\PY{l+m+mi}{1}\PY{p}{]}\PY{p}{,}\PY{+w}{ }\PY{n}{p}\PY{p}{[}\PY{l+m+mi}{2}\PY{p}{]}\PY{p}{,}\PY{+w}{ }\PY{n}{p}\PY{p}{[}\PY{l+m+mi}{3}\PY{p}{]}\PY{p}{)}\PY{p}{,}
\PY{n}{p}\PY{+w}{ }\PY{o}{\PYZhy{}\PYZgt{}}\PY{+w}{ }\PY{n}{propose}\PY{p}{(}\PY{n}{p}\PY{p}{,}\PY{+w}{ }\PY{n}{jump\PYZus{}sizes}\PY{p}{)}\PY{p}{,}
\PY{n}{initial\PYZus{}guess}\PY{p}{,}\PY{+w}{ }\PY{l+m+mi}{150000}\PY{p}{)}
\PY{n}{println}\PY{p}{(}\PY{l+s}{\PYZdq{}}\PY{l+s}{Acceptance Rates}\PY{l+s}{\PYZdq{}}\PY{p}{)}
\PY{n}{accept\PYZus{}rate}
\end{Verbatim}
\end{tcolorbox}

    \begin{Verbatim}[commandchars=\\\{\}]
Acceptance Rates
    \end{Verbatim}

            \begin{tcolorbox}[breakable, size=fbox, boxrule=.5pt, pad at break*=1mm, opacityfill=0]
\prompt{Out}{outcolor}{15}{\boxspacing}
\begin{Verbatim}[commandchars=\\\{\}]
3-element Vector\{Float64\}:
 0.78388
 0.69756
 0.71164
\end{Verbatim}
\end{tcolorbox}
        
    I only plot the samples as the corner plot as it makes the sampling most
clear.

    \begin{tcolorbox}[breakable, size=fbox, boxrule=1pt, pad at break*=1mm,colback=cellbackground, colframe=cellborder]
\prompt{In}{incolor}{29}{\boxspacing}
\begin{Verbatim}[commandchars=\\\{\}]
\PY{n}{cplot1}\PY{+w}{ }\PY{o}{=}\PY{+w}{ }\PY{n}{cornerplot}\PY{p}{(}\PY{n}{chain}\PY{p}{,}\PY{+w}{ }\PY{p}{[}\PY{l+s}{\PYZdq{}}\PY{l+s}{M}\PY{l+s}{\PYZdq{}}\PY{p}{,}\PY{+w}{ }\PY{l+s}{\PYZdq{}}\PY{l+s}{Omega\PYZus{}M}\PY{l+s}{\PYZdq{}}\PY{p}{,}\PY{+w}{ }\PY{l+s}{\PYZdq{}}\PY{l+s}{Omega\PYZus{}DE}\PY{l+s}{\PYZdq{}}\PY{p}{]}\PY{p}{)}
\PY{n}{save}\PY{p}{(}\PY{l+s}{\PYZdq{}}\PY{l+s}{cornerplotFirstMCMC.png}\PY{l+s}{\PYZdq{}}\PY{p}{,}\PY{+w}{ }\PY{n}{cplot1}\PY{p}{)}
\end{Verbatim}
\end{tcolorbox}

    \begin{center}
      
      \includegraphics[]{cornerplotFirstMCMC.png}
    \end{center}

    
    \begin{Verbatim}[commandchars=\\\{\}]
Bonito.App(Bonito.var"\#8\#14"\{WGLMakie.var"\#20\#21"\{WGLMakie.Screen, Scene\}\}(WGLMakie.var"\#20\#21"\{WGLMakie.Screen, Scene\}(WGLMakie.Screen(
    framerate = 30.0,
    resize\_to = nothing,
    px\_per\_unit = automatic,
    scalefactor = automatic,
    session = nothing,
    three = Channel\{Bool\}(1) (empty),
    scene = Scene (600px, 600px):
  0 Plots
  6 Child Scenes:
    ├ Scene (600px, 600px)
    ├ Scene (600px, 600px)
    ├ Scene (600px, 600px)
    ├ Scene (600px, 600px)
    ├ Scene (600px, 600px)
    └ Scene (600px, 600px),
), Scene (600px, 600px):
  0 Plots
  6 Child Scenes:
    ├ Scene (600px, 600px)
    ├ Scene (600px, 600px)
    ├ Scene (600px, 600px)
    ├ Scene (600px, 600px)
    ├ Scene (600px, 600px)
    └ Scene (600px, 600px))), Base.RefValue\{Union\{Nothing, Bonito.Session\}\}(nothing), "Bonito App", false)
    \end{Verbatim}

    
    From the above corner plot, \(\Omega_M\) and \(M\) seem slightly
negatively correlated, \(\Omega_{DE}\) and \(M\) also seem slightly
negatively correlated. Finally, \(\Omega_M\) and \(\Omega_{DE}\) are
strongly positively correlated.

    \begin{tcolorbox}[breakable, size=fbox, boxrule=1pt, pad at break*=1mm,colback=cellbackground, colframe=cellborder]
\prompt{In}{incolor}{17}{\boxspacing}
\begin{Verbatim}[commandchars=\\\{\}]
\PY{n}{f}\PY{+w}{ }\PY{o}{=}\PY{+w}{ }\PY{n}{Figure}\PY{p}{(}\PY{n}{size}\PY{+w}{ }\PY{o}{=}\PY{+w}{ }\PY{p}{(}\PY{l+m+mi}{1500}\PY{p}{,}\PY{+w}{ }\PY{l+m+mi}{400}\PY{p}{)}\PY{p}{)}
\PY{n}{ax}\PY{+w}{ }\PY{o}{=}\PY{+w}{ }\PY{n}{Axis}\PY{p}{(}\PY{n}{f}\PY{p}{[}\PY{l+m+mi}{1}\PY{p}{,}\PY{l+m+mi}{1}\PY{p}{]}\PY{p}{,}\PY{+w}{ }\PY{n}{title}\PY{o}{=}\PY{l+s}{\PYZdq{}}\PY{l+s}{MCMC chain for M}\PY{l+s}{\PYZdq{}}\PY{p}{,}\PY{+w}{ }\PY{n}{xlabel}\PY{o}{=}\PY{l+s}{\PYZdq{}}\PY{l+s}{step}\PY{l+s}{\PYZdq{}}\PY{p}{,}\PY{+w}{ }\PY{n}{ylabel}\PY{o}{=}\PY{l+s}{\PYZdq{}}\PY{l+s}{M}\PY{l+s}{\PYZdq{}}\PY{p}{)}

\PY{n}{scatter!}\PY{p}{(}\PY{n}{chain}\PY{p}{[}\PY{l+m+mi}{1}\PY{o}{:}\PY{l+m+mi}{10}\PY{o}{:}\PY{k}{end}\PY{p}{,}\PY{+w}{ }\PY{l+m+mi}{1}\PY{p}{]}\PY{p}{)}
\PY{n}{ax}\PY{+w}{ }\PY{o}{=}\PY{+w}{ }\PY{n}{Axis}\PY{p}{(}\PY{n}{f}\PY{p}{[}\PY{l+m+mi}{1}\PY{p}{,}\PY{l+m+mi}{2}\PY{p}{]}\PY{p}{,}\PY{+w}{ }\PY{n}{title}\PY{o}{=}\PY{l+s}{\PYZdq{}}\PY{l+s}{MCMC chain for Omega\PYZus{}M}\PY{l+s}{\PYZdq{}}\PY{p}{,}\PY{+w}{ }\PY{n}{xlabel}\PY{o}{=}\PY{l+s}{\PYZdq{}}\PY{l+s}{step}\PY{l+s}{\PYZdq{}}\PY{p}{,}\PY{+w}{ }\PY{n}{ylabel}\PY{o}{=}\PY{l+s}{\PYZdq{}}\PY{l+s}{Omega\PYZus{}M}\PY{l+s}{\PYZdq{}}\PY{p}{)}
\PY{n}{scatter!}\PY{p}{(}\PY{n}{chain}\PY{p}{[}\PY{l+m+mi}{1}\PY{o}{:}\PY{l+m+mi}{10}\PY{o}{:}\PY{k}{end}\PY{p}{,}\PY{+w}{ }\PY{l+m+mi}{2}\PY{p}{]}\PY{p}{)}
\PY{n}{ax}\PY{+w}{ }\PY{o}{=}\PY{+w}{ }\PY{n}{Axis}\PY{p}{(}\PY{n}{f}\PY{p}{[}\PY{l+m+mi}{1}\PY{p}{,}\PY{l+m+mi}{3}\PY{p}{]}\PY{p}{,}\PY{+w}{ }\PY{n}{title}\PY{o}{=}\PY{l+s}{\PYZdq{}}\PY{l+s}{MCMC chain for Omega\PYZus{}DE}\PY{l+s}{\PYZdq{}}\PY{p}{,}\PY{+w}{ }\PY{n}{xlabel}\PY{o}{=}\PY{l+s}{\PYZdq{}}\PY{l+s}{step}\PY{l+s}{\PYZdq{}}\PY{p}{,}\PY{+w}{ }\PY{n}{ylabel}\PY{o}{=}\PY{l+s}{\PYZdq{}}\PY{l+s}{Omega\PYZus{}DE}\PY{l+s}{\PYZdq{}}\PY{p}{)}
\PY{n}{scatter!}\PY{p}{(}\PY{n}{chain}\PY{p}{[}\PY{l+m+mi}{1}\PY{o}{:}\PY{l+m+mi}{10}\PY{o}{:}\PY{k}{end}\PY{p}{,}\PY{+w}{ }\PY{l+m+mi}{3}\PY{p}{]}\PY{p}{)}

\PY{n}{save}\PY{p}{(}\PY{l+s}{\PYZdq{}}\PY{l+s}{MCMCChain.png}\PY{l+s}{\PYZdq{}}\PY{p}{,}\PY{+w}{ }\PY{n}{f}\PY{p}{)}
\PY{n}{display}\PY{p}{(}\PY{n}{f}\PY{p}{)}
\end{Verbatim}
\end{tcolorbox}

    \begin{center}
      
      \includegraphics[scale=1.1]{MCMCChain.png}
    \end{center}

    
    \begin{Verbatim}[commandchars=\\\{\}]
Bonito.App(Bonito.var"\#8\#14"\{WGLMakie.var"\#20\#21"\{WGLMakie.Screen, Scene\}\}(WGLMakie.var"\#20\#21"\{WGLMakie.Screen, Scene\}(WGLMakie.Screen(
    framerate = 30.0,
    resize\_to = nothing,
    px\_per\_unit = automatic,
    scalefactor = automatic,
    session = nothing,
    three = Channel\{Bool\}(1) (empty),
    scene = Scene (1500px, 400px):
  0 Plots
  3 Child Scenes:
    ├ Scene (1500px, 400px)
    ├ Scene (1500px, 400px)
    └ Scene (1500px, 400px),
), Scene (1500px, 400px):
  0 Plots
  3 Child Scenes:
    ├ Scene (1500px, 400px)
    ├ Scene (1500px, 400px)
    └ Scene (1500px, 400px))), Base.RefValue\{Union\{Nothing, Bonito.Session\}\}(nothing), "Bonito App", false)
    \end{Verbatim}

    
    
    
            \begin{tcolorbox}[breakable, size=fbox, boxrule=.5pt, pad at break*=1mm, opacityfill=0]
\prompt{Out}{outcolor}{17}{\boxspacing}
\begin{Verbatim}[commandchars=\\\{\}]
WGLMakie.Screen(
    framerate = 30.0,
    resize\_to = nothing,
    px\_per\_unit = automatic,
    scalefactor = automatic,
    session = true,
    three = Channel\{Bool\}(1) (1 item available),
    scene = Scene (1500px, 400px):
  0 Plots
  3 Child Scenes:
    ├ Scene (1500px, 400px)
    ├ Scene (1500px, 400px)
    └ Scene (1500px, 400px),
)
\end{Verbatim}
\end{tcolorbox}
        
    Finally, please try to make a contour plot similar to Perlmutter et
al.'s Figure 7. From your MCMC chain, you can pull out the
\texttt{Omega\_M} and \texttt{Omega\_DE} arrays, and then create a 2-d
histogram. Once you have a 2-d histogram, you can use the
\texttt{contour} function to find and plot the contours in that
histogram.

    \begin{tcolorbox}[breakable, size=fbox, boxrule=1pt, pad at break*=1mm,colback=cellbackground, colframe=cellborder]
\prompt{In}{incolor}{18}{\boxspacing}
\begin{Verbatim}[commandchars=\\\{\}]
\PY{n}{OmegaMMCMC}\PY{+w}{ }\PY{o}{=}\PY{+w}{ }\PY{n}{chain}\PY{p}{[}\PY{n}{begin}\PY{o}{:}\PY{n}{end}\PY{p}{,}\PY{+w}{ }\PY{l+m+mi}{2}\PY{p}{]}
\PY{n}{OmegaDEMCMC}\PY{+w}{ }\PY{o}{=}\PY{+w}{ }\PY{n}{chain}\PY{p}{[}\PY{n}{begin}\PY{o}{:}\PY{n}{end}\PY{p}{,}\PY{+w}{ }\PY{l+m+mi}{3}\PY{p}{]}\PY{p}{;}
\end{Verbatim}
\end{tcolorbox}

    Here I bin the data in a 2D histogram. I did it in a for loop placing
the histogram values in appropriate bins.

    \begin{tcolorbox}[breakable, size=fbox, boxrule=1pt, pad at break*=1mm,colback=cellbackground, colframe=cellborder]
\prompt{In}{incolor}{19}{\boxspacing}
\begin{Verbatim}[commandchars=\\\{\}]
\PY{n}{n\PYZus{}ombins}\PY{p}{,}\PY{+w}{ }\PY{n}{n\PYZus{}odebins}\PY{+w}{ }\PY{o}{=}\PY{+w}{ }\PY{l+m+mi}{20}\PY{p}{,}\PY{l+m+mi}{20}

\PY{n}{om\PYZus{}gridMCMC}\PY{+w}{ }\PY{o}{=}\PY{+w}{ }\PY{k+kt}{LinRange}\PY{p}{(}\PY{n}{minimum}\PY{p}{(}\PY{n}{OmegaMMCMC}\PY{p}{)}\PY{p}{,}\PY{+w}{ }\PY{n}{maximum}\PY{p}{(}\PY{n}{OmegaMMCMC}\PY{p}{)}\PY{p}{,}\PY{+w}{ }\PY{n}{n\PYZus{}ombins}\PY{+w}{ }\PY{o}{+}\PY{+w}{ }\PY{l+m+mi}{1}\PY{p}{)}
\PY{n}{ode\PYZus{}gridMCMC}\PY{+w}{ }\PY{o}{=}\PY{+w}{ }\PY{k+kt}{LinRange}\PY{p}{(}\PY{n}{minimum}\PY{p}{(}\PY{n}{OmegaDEMCMC}\PY{p}{)}\PY{p}{,}\PY{+w}{ }\PY{n}{maximum}\PY{p}{(}\PY{n}{OmegaDEMCMC}\PY{p}{)}\PY{p}{,}\PY{+w}{ }\PY{n}{n\PYZus{}odebins}\PY{+w}{ }\PY{o}{+}\PY{+w}{ }\PY{l+m+mi}{1}\PY{p}{)}

\PY{n}{minOmegaMMCMC}\PY{+w}{ }\PY{o}{=}\PY{+w}{ }\PY{n}{minimum}\PY{p}{(}\PY{n}{OmegaMMCMC}\PY{p}{)}
\PY{n}{minOmegaDEMCMC}\PY{+w}{ }\PY{o}{=}\PY{+w}{ }\PY{n}{minimum}\PY{p}{(}\PY{n}{OmegaDEMCMC}\PY{p}{)}
\PY{n}{maxOmegaMMCMC}\PY{+w}{ }\PY{o}{=}\PY{+w}{ }\PY{n}{maximum}\PY{p}{(}\PY{n}{OmegaMMCMC}\PY{p}{)}
\PY{n}{maxOmegaDEMCMC}\PY{+w}{ }\PY{o}{=}\PY{+w}{ }\PY{n}{maximum}\PY{p}{(}\PY{n}{OmegaDEMCMC}\PY{p}{)}
\PY{n}{rangeOmegaMMCMC}\PY{+w}{ }\PY{o}{=}\PY{+w}{ }\PY{n}{maxOmegaMMCMC}\PY{+w}{ }\PY{o}{\PYZhy{}}\PY{+w}{ }\PY{n}{minOmegaMMCMC}
\PY{n}{rangeOmegaDEMCMC}\PY{+w}{ }\PY{o}{=}\PY{+w}{ }\PY{n}{maxOmegaDEMCMC}\PY{+w}{ }\PY{o}{\PYZhy{}}\PY{+w}{ }\PY{n}{minOmegaDEMCMC}

\PY{n}{histogramValues}\PY{+w}{ }\PY{o}{=}\PY{+w}{ }\PY{n}{zeros}\PY{p}{(}\PY{n}{n\PYZus{}ombins}\PY{+w}{ }\PY{o}{+}\PY{+w}{ }\PY{l+m+mi}{1}\PY{p}{,}\PY{+w}{ }\PY{n}{n\PYZus{}odebins}\PY{+w}{ }\PY{o}{+}\PY{+w}{ }\PY{l+m+mi}{1}\PY{p}{)}
\PY{k}{for}\PY{+w}{ }\PY{n}{i}\PY{+w}{ }\PY{k}{in}\PY{+w}{ }\PY{l+m+mi}{1}\PY{o}{:}\PY{n}{size}\PY{p}{(}\PY{n}{OmegaDEMCMC}\PY{p}{)}\PY{p}{[}\PY{k}{begin}\PY{p}{]}
\PY{+w}{    }\PY{n}{histogramValues}\PY{p}{[}\PY{n}{trunc}\PY{p}{(}\PY{k+kt}{Int}\PY{p}{,}\PY{+w}{ }\PY{p}{(}\PY{n}{OmegaMMCMC}\PY{p}{[}\PY{n}{i}\PY{p}{]}\PY{+w}{ }\PY{o}{\PYZhy{}}\PY{+w}{ }\PY{n}{minOmegaMMCMC}\PY{p}{)}\PY{+w}{ }\PY{o}{*}\PY{+w}{ }\PY{n}{n\PYZus{}ombins}\PY{+w}{ }\PY{o}{/}\PY{+w}{ }\PY{p}{(}\PY{n}{rangeOmegaMMCMC}\PY{p}{)}\PY{p}{)}\PY{+w}{ }\PY{o}{+}\PY{+w}{ }\PY{l+m+mi}{1}\PY{p}{,}\PY{+w}{ }\PY{n}{trunc}\PY{p}{(}\PY{k+kt}{Int}\PY{p}{,}\PY{+w}{ }\PY{p}{(}\PY{n}{OmegaDEMCMC}\PY{p}{[}\PY{n}{i}\PY{p}{]}\PY{+w}{ }\PY{o}{\PYZhy{}}\PY{+w}{ }\PY{n}{minOmegaDEMCMC}\PY{p}{)}\PY{+w}{ }\PY{o}{*}\PY{+w}{ }\PY{n}{n\PYZus{}odebins}\PY{+w}{ }\PY{o}{/}\PY{+w}{ }\PY{n}{rangeOmegaDEMCMC}\PY{p}{)}\PY{+w}{ }\PY{o}{+}\PY{+w}{ }\PY{l+m+mi}{1}\PY{p}{]}\PY{+w}{ }\PY{o}{+=}\PY{+w}{ }\PY{l+m+mi}{1}
\PY{c}{\PYZsh{}\PYZsh{} I was doing to more manually, which is probably correct, but very slow.}
\PY{c}{\PYZsh{}    for j in 1:(n\PYZus{}ombins \PYZhy{} 1)}
\PY{c}{\PYZsh{}        for k in 1:(n\PYZus{}odebins \PYZhy{} 1)}
\PY{c}{\PYZsh{}            histogramValues[j, k] += (om\PYZus{}gridMCMC[j] \PYZlt{}= OmegaMMCMC[i] \PYZam{}\PYZam{} OmegaMMCMC[i] \PYZlt{}= om\PYZus{}gridMCMC[j + 1] \PYZam{}\PYZam{} ode\PYZus{}gridMCMC[k] \PYZlt{}= OmegaDEMCMC[i] \PYZam{}\PYZam{} OmegaDEMCMC[i] \PYZlt{}= ode\PYZus{}gridMCMC[k + 1])}
\PY{c}{\PYZsh{}        end}
\PY{c}{\PYZsh{}    end}
\PY{k}{end}
\PY{n}{histogramValues}\PY{p}{;}
\end{Verbatim}
\end{tcolorbox}

    Next I just call the contour plot function on the histogram matrix.

    \begin{tcolorbox}[breakable, size=fbox, boxrule=1pt, pad at break*=1mm,colback=cellbackground, colframe=cellborder]
\prompt{In}{incolor}{20}{\boxspacing}
\begin{Verbatim}[commandchars=\\\{\}]
\PY{n}{f}\PY{+w}{ }\PY{o}{=}\PY{+w}{ }\PY{n}{Figure}\PY{p}{(}\PY{p}{)}
\PY{n}{Axis}\PY{p}{(}\PY{n}{f}\PY{p}{[}\PY{l+m+mi}{1}\PY{p}{,}\PY{+w}{ }\PY{l+m+mi}{1}\PY{p}{]}\PY{p}{)}

\PY{n}{contour!}\PY{p}{(}\PY{n}{om\PYZus{}gridMCMC}\PY{p}{,}\PY{+w}{ }\PY{n}{ode\PYZus{}gridMCMC}\PY{p}{,}\PY{+w}{ }\PY{n}{histogramValues}\PY{p}{)}

\PY{n}{save}\PY{p}{(}\PY{l+s}{\PYZdq{}}\PY{l+s}{MyHistogram.png}\PY{l+s}{\PYZdq{}}\PY{p}{,}\PY{+w}{ }\PY{n}{f}\PY{p}{)}

\PY{n}{display}\PY{p}{(}\PY{n}{f}\PY{p}{)}
\end{Verbatim}
\end{tcolorbox}

    \begin{center}
      
      \includegraphics[]{MyHistogram.png}
    \end{center}

    
    \begin{Verbatim}[commandchars=\\\{\}]
Bonito.App(Bonito.var"\#8\#14"\{WGLMakie.var"\#20\#21"\{WGLMakie.Screen, Scene\}\}(WGLMakie.var"\#20\#21"\{WGLMakie.Screen, Scene\}(WGLMakie.Screen(
    framerate = 30.0,
    resize\_to = nothing,
    px\_per\_unit = automatic,
    scalefactor = automatic,
    session = nothing,
    three = Channel\{Bool\}(1) (empty),
    scene = Scene (600px, 450px):
  0 Plots
  1 Child Scene:
    └ Scene (600px, 450px),
), Scene (600px, 450px):
  0 Plots
  1 Child Scene:
    └ Scene (600px, 450px))), Base.RefValue\{Union\{Nothing, Bonito.Session\}\}(nothing), "Bonito App", false)
    \end{Verbatim}

    
    
    
            \begin{tcolorbox}[breakable, size=fbox, boxrule=.5pt, pad at break*=1mm, opacityfill=0]
\prompt{Out}{outcolor}{20}{\boxspacing}
\begin{Verbatim}[commandchars=\\\{\}]
WGLMakie.Screen(
    framerate = 30.0,
    resize\_to = nothing,
    px\_per\_unit = automatic,
    scalefactor = automatic,
    session = true,
    three = Channel\{Bool\}(1) (1 item available),
    scene = Scene (600px, 450px):
  0 Plots
  1 Child Scene:
    └ Scene (600px, 450px),
)
\end{Verbatim}
\end{tcolorbox}
        
    \begin{tcolorbox}[breakable, size=fbox, boxrule=1pt, pad at break*=1mm,colback=cellbackground, colframe=cellborder]
\prompt{In}{incolor}{21}{\boxspacing}
\begin{Verbatim}[commandchars=\\\{\}]
\PY{p}{]}\PY{+w}{ }\PY{n}{add}\PY{+w}{ }\PY{n}{FHist}\PY{+w}{ }\PY{n}{AffineInvariantMCMC}
\end{Verbatim}
\end{tcolorbox}

    \begin{Verbatim}[commandchars=\\\{\}]
\textcolor{ansi-green-intense}{\textbf{   Resolving}} package versions{\ldots}
\textcolor{ansi-green-intense}{\textbf{  No Changes}} to `\textasciitilde{}/.julia/environments/v1.10/Project.toml`
\textcolor{ansi-green-intense}{\textbf{  No Changes}} to `\textasciitilde{}/.julia/environments/v1.10/Manifest.toml`
    \end{Verbatim}

    Here I tested the FHist package that was mentioned in the lectures.

    \begin{tcolorbox}[breakable, size=fbox, boxrule=1pt, pad at break*=1mm,colback=cellbackground, colframe=cellborder]
\prompt{In}{incolor}{22}{\boxspacing}
\begin{Verbatim}[commandchars=\\\{\}]
\PY{k}{using}\PY{+w}{ }\PY{n}{FHist}

\PY{n}{h}\PY{+w}{ }\PY{o}{=}\PY{+w}{ }\PY{n}{FHist}\PY{o}{.}\PY{n}{Hist2D}\PY{p}{(}\PY{p}{(}\PY{n}{chain}\PY{p}{[}\PY{o}{:}\PY{p}{,}\PY{l+m+mi}{2}\PY{p}{]}\PY{p}{,}\PY{+w}{ }\PY{n}{chain}\PY{p}{[}\PY{o}{:}\PY{p}{,}\PY{l+m+mi}{3}\PY{p}{]}\PY{p}{)}\PY{p}{;}\PY{+w}{ }\PY{n}{nbins}\PY{o}{=}\PY{p}{(}\PY{l+m+mi}{100}\PY{p}{,}\PY{l+m+mi}{100}\PY{p}{)}\PY{p}{)}

\PY{n}{counts}\PY{+w}{ }\PY{o}{=}\PY{+w}{ }\PY{n}{bincounts}\PY{p}{(}\PY{n}{h}\PY{p}{)}\PY{p}{;}
\PY{n}{xc}\PY{p}{,}\PY{n}{yc}\PY{+w}{ }\PY{o}{=}\PY{+w}{ }\PY{n}{bincenters}\PY{p}{(}\PY{n}{h}\PY{p}{)}\PY{p}{;}

\PY{n}{f}\PY{+w}{ }\PY{o}{=}\PY{+w}{ }\PY{n}{Figure}\PY{p}{(}\PY{p}{)}
\PY{n}{Axis}\PY{p}{(}\PY{n}{f}\PY{p}{[}\PY{l+m+mi}{1}\PY{p}{,}\PY{+w}{ }\PY{l+m+mi}{1}\PY{p}{]}\PY{p}{)}

\PY{n}{contour!}\PY{p}{(}\PY{n}{xc}\PY{p}{,}\PY{+w}{ }\PY{n}{yc}\PY{p}{,}\PY{+w}{ }\PY{n}{counts}\PY{p}{)}\PY{+w}{ }\PY{c}{\PYZsh{}, levels=[10,50,100])}
\PY{c}{\PYZsh{}really small/many bins make this contour plot really jaggered}

\PY{n}{save}\PY{p}{(}\PY{l+s}{\PYZdq{}}\PY{l+s}{MyFirstJaggeredContour.png}\PY{l+s}{\PYZdq{}}\PY{p}{,}\PY{+w}{ }\PY{n}{f}\PY{p}{)}
\PY{n}{f}
\end{Verbatim}
\end{tcolorbox}

    \begin{center}
      
      \includegraphics[]{MyFirstJaggeredContour.png}
    \end{center}

    
    \begin{Verbatim}[commandchars=\\\{\}]
Bonito.App(Bonito.var"\#8\#14"\{WGLMakie.var"\#20\#21"\{WGLMakie.Screen, Scene\}\}(WGLMakie.var"\#20\#21"\{WGLMakie.Screen, Scene\}(WGLMakie.Screen(
    framerate = 30.0,
    resize\_to = nothing,
    px\_per\_unit = automatic,
    scalefactor = automatic,
    session = nothing,
    three = Channel\{Bool\}(1) (empty),
    scene = Scene (600px, 450px):
  0 Plots
  1 Child Scene:
    └ Scene (600px, 450px),
), Scene (600px, 450px):
  0 Plots
  1 Child Scene:
    └ Scene (600px, 450px))), Base.RefValue\{Union\{Nothing, Bonito.Session\}\}(nothing), "Bonito App", false)
    \end{Verbatim}

    
    Doing this with fewer bigger bins makes the contours smoother, similar
to my manual way. FHist seems to be binning data slightly differently to
me\ldots{}

    \begin{tcolorbox}[breakable, size=fbox, boxrule=1pt, pad at break*=1mm,colback=cellbackground, colframe=cellborder]
\prompt{In}{incolor}{23}{\boxspacing}
\begin{Verbatim}[commandchars=\\\{\}]
\PY{n}{h}\PY{+w}{ }\PY{o}{=}\PY{+w}{ }\PY{n}{FHist}\PY{o}{.}\PY{n}{Hist2D}\PY{p}{(}\PY{p}{(}\PY{n}{chain}\PY{p}{[}\PY{o}{:}\PY{p}{,}\PY{l+m+mi}{2}\PY{p}{]}\PY{p}{,}\PY{+w}{ }\PY{n}{chain}\PY{p}{[}\PY{o}{:}\PY{p}{,}\PY{l+m+mi}{3}\PY{p}{]}\PY{p}{)}\PY{p}{;}\PY{+w}{ }\PY{n}{nbins}\PY{o}{=}\PY{p}{(}\PY{l+m+mi}{20}\PY{p}{,}\PY{l+m+mi}{20}\PY{p}{)}\PY{p}{)}

\PY{n}{counts}\PY{+w}{ }\PY{o}{=}\PY{+w}{ }\PY{n}{bincounts}\PY{p}{(}\PY{n}{h}\PY{p}{)}\PY{p}{;}
\PY{n}{xc}\PY{p}{,}\PY{n}{yc}\PY{+w}{ }\PY{o}{=}\PY{+w}{ }\PY{n}{bincenters}\PY{p}{(}\PY{n}{h}\PY{p}{)}\PY{p}{;}

\PY{n}{f}\PY{+w}{ }\PY{o}{=}\PY{+w}{ }\PY{n}{Figure}\PY{p}{(}\PY{p}{)}
\PY{n}{Axis}\PY{p}{(}\PY{n}{f}\PY{p}{[}\PY{l+m+mi}{1}\PY{p}{,}\PY{+w}{ }\PY{l+m+mi}{1}\PY{p}{]}\PY{p}{)}

\PY{n}{contour!}\PY{p}{(}\PY{n}{xc}\PY{p}{,}\PY{+w}{ }\PY{n}{yc}\PY{p}{,}\PY{+w}{ }\PY{n}{counts}\PY{p}{)}\PY{c}{\PYZsh{}, levels=[10,50,100])}

\PY{n}{save}\PY{p}{(}\PY{l+s}{\PYZdq{}}\PY{l+s}{SmallerBinsCountour.png}\PY{l+s}{\PYZdq{}}\PY{p}{,}\PY{+w}{ }\PY{n}{f}\PY{p}{)}
\PY{n}{f}
\end{Verbatim}
\end{tcolorbox}

    \begin{center}
      
      \includegraphics[]{SmallerBinsCountour.png}
    \end{center}

    
    \begin{Verbatim}[commandchars=\\\{\}]
Bonito.App(Bonito.var"\#8\#14"\{WGLMakie.var"\#20\#21"\{WGLMakie.Screen, Scene\}\}(WGLMakie.var"\#20\#21"\{WGLMakie.Screen, Scene\}(WGLMakie.Screen(
    framerate = 30.0,
    resize\_to = nothing,
    px\_per\_unit = automatic,
    scalefactor = automatic,
    session = nothing,
    three = Channel\{Bool\}(1) (empty),
    scene = Scene (600px, 450px):
  0 Plots
  1 Child Scene:
    └ Scene (600px, 450px),
), Scene (600px, 450px):
  0 Plots
  1 Child Scene:
    └ Scene (600px, 450px))), Base.RefValue\{Union\{Nothing, Bonito.Session\}\}(nothing), "Bonito App", false)
    \end{Verbatim}

    
    Doing samples with the Affine MCMC method mentioned in the lectures.
``Self-tunes parameters''.

    \begin{tcolorbox}[breakable, size=fbox, boxrule=1pt, pad at break*=1mm,colback=cellbackground, colframe=cellborder]
\prompt{In}{incolor}{24}{\boxspacing}
\begin{Verbatim}[commandchars=\\\{\}]
\PY{k}{using}\PY{+w}{ }\PY{n}{AffineInvariantMCMC}

\PY{n}{numdims}\PY{+w}{ }\PY{o}{=}\PY{+w}{ }\PY{l+m+mi}{3}
\PY{n}{numwalkers}\PY{+w}{ }\PY{o}{=}\PY{+w}{ }\PY{l+m+mi}{50}
\PY{n}{thinning}\PY{+w}{ }\PY{o}{=}\PY{+w}{ }\PY{l+m+mi}{10}
\PY{n}{numsamples\PYZus{}perwalker}\PY{+w}{ }\PY{o}{=}\PY{+w}{ }\PY{l+m+mi}{15000}
\PY{n}{burnin}\PY{+w}{ }\PY{o}{=}\PY{+w}{ }\PY{l+m+mi}{5000}

\PY{c}{\PYZsh{} Start out by doing a \PYZdq{}burn\PYZhy{}in\PYZdq{} short run...}
\PY{n}{initial}\PY{+w}{ }\PY{o}{=}\PY{+w}{ }\PY{p}{[}\PY{n}{abs\PYZus{}mag}\PY{p}{,}\PY{+w}{ }\PY{l+m+mf}{0.5}\PY{p}{,}\PY{+w}{ }\PY{l+m+mf}{0.5}\PY{p}{]}\PY{+w}{ }\PY{o}{.+}\PY{+w}{ }\PY{n}{randn}\PY{p}{(}\PY{n}{numdims}\PY{p}{,}\PY{+w}{ }\PY{n}{numwalkers}\PY{p}{)}\PY{o}{*}\PY{l+m+mf}{0.01}
\PY{n}{chain}\PY{p}{,}\PY{+w}{ }\PY{n}{ll}\PY{+w}{ }\PY{o}{=}\PY{+w}{ }\PY{n}{AffineInvariantMCMC}\PY{o}{.}\PY{n}{sample}\PY{p}{(}\PY{+w}{ }\PY{n}{p}\PY{+w}{ }\PY{o}{\PYZhy{}\PYZgt{}}\PY{+w}{ }
\PY{+w}{        }\PY{k}{try}
\PY{+w}{            }\PY{n}{supernova\PYZus{}log\PYZus{}likelihood}\PY{p}{(}\PY{n}{data}\PY{o}{.}\PY{n}{z}\PY{p}{,}\PY{+w}{ }\PY{n}{data}\PY{o}{.}\PY{n}{mag}\PY{p}{,}\PY{+w}{ }\PY{n}{data}\PY{o}{.}\PY{n}{sigma\PYZus{}mag}\PY{p}{,}\PY{+w}{ }\PY{n}{p}\PY{p}{[}\PY{l+m+mi}{1}\PY{p}{]}\PY{p}{,}\PY{+w}{ }\PY{n}{p}\PY{p}{[}\PY{l+m+mi}{2}\PY{p}{]}\PY{p}{,}\PY{+w}{ }\PY{n}{p}\PY{p}{[}\PY{l+m+mi}{3}\PY{p}{]}\PY{p}{)}
\PY{+w}{        }\PY{k}{catch}\PY{+w}{ }\PY{n}{err}
\PY{+w}{            }\PY{o}{\PYZhy{}}\PY{n+nb}{Inf}
\PY{+w}{        }\PY{k}{end}\PY{p}{,}\PY{+w}{ }\PY{n}{numwalkers}\PY{p}{,}\PY{+w}{ }\PY{n}{initial}\PY{p}{,}\PY{+w}{ }\PY{n}{burnin}\PY{p}{,}\PY{+w}{ }\PY{l+m+mi}{1}\PY{p}{)}

\PY{c}{\PYZsh{} Then start the \PYZdq{}main\PYZdq{} run where the burn\PYZhy{}in finished:}
\PY{n}{initial}\PY{+w}{ }\PY{o}{=}\PY{+w}{ }\PY{n}{chain}\PY{p}{[}\PY{o}{:}\PY{p}{,}\PY{o}{:}\PY{p}{,}\PY{k}{end}\PY{p}{]}
\PY{n}{chain}\PY{p}{,}\PY{+w}{ }\PY{n}{ll}\PY{+w}{ }\PY{o}{=}\PY{+w}{ }\PY{n}{AffineInvariantMCMC}\PY{o}{.}\PY{n}{sample}\PY{p}{(}\PY{+w}{ }
\PY{+w}{   }\PY{n}{p}\PY{+w}{ }\PY{o}{\PYZhy{}\PYZgt{}}\PY{+w}{ }
\PY{+w}{        }\PY{k}{try}
\PY{+w}{            }\PY{n}{supernova\PYZus{}log\PYZus{}likelihood}\PY{p}{(}\PY{n}{data}\PY{o}{.}\PY{n}{z}\PY{p}{,}\PY{+w}{ }\PY{n}{data}\PY{o}{.}\PY{n}{mag}\PY{p}{,}\PY{+w}{ }\PY{n}{data}\PY{o}{.}\PY{n}{sigma\PYZus{}mag}\PY{p}{,}\PY{+w}{ }\PY{n}{p}\PY{p}{[}\PY{l+m+mi}{1}\PY{p}{]}\PY{p}{,}\PY{+w}{ }\PY{n}{p}\PY{p}{[}\PY{l+m+mi}{2}\PY{p}{]}\PY{p}{,}\PY{+w}{ }\PY{n}{p}\PY{p}{[}\PY{l+m+mi}{3}\PY{p}{]}\PY{p}{)}
\PY{+w}{        }\PY{k}{catch}\PY{+w}{ }\PY{n}{err}
\PY{+w}{            }\PY{o}{\PYZhy{}}\PY{n+nb}{Inf}
\PY{+w}{        }\PY{k}{end}
\PY{+w}{    }\PY{p}{,}\PY{+w}{ }\PY{n}{numwalkers}\PY{p}{,}\PY{+w}{ }\PY{n}{initial}\PY{p}{,}\PY{+w}{ }\PY{n}{numsamples\PYZus{}perwalker}\PY{p}{,}\PY{+w}{ }\PY{n}{thinning}\PY{p}{)}
\PY{n}{flatchain}\PY{p}{,}\PY{+w}{ }\PY{n}{flatllhoodvals}\PY{+w}{ }\PY{o}{=}\PY{+w}{ }\PY{n}{AffineInvariantMCMC}\PY{o}{.}\PY{+w}{ }\PY{n}{flattenmcmcarray}\PY{p}{(}\PY{n}{chain}\PY{p}{,}\PY{+w}{ }\PY{n}{ll}\PY{p}{)}\PY{p}{;}

\PY{c}{\PYZsh{} This \PYZdq{}flatchain\PYZdq{} is transposed from the way we produced it in our code, so you can do}

\PY{n}{chain}\PY{+w}{ }\PY{o}{=}\PY{+w}{ }\PY{n}{flatchain}\PY{o}{\PYZsq{}}\PY{p}{;}
\end{Verbatim}
\end{tcolorbox}

    \begin{Verbatim}[commandchars=\\\{\}]
\textcolor{ansi-green}{Progress: 100\%|█████████████████████████████████████████| Time:
0:00:12}
    \end{Verbatim}

    The affine MCMC doesn't seem to give perfect sampling, some samples seem
to be out of the main distribution, even starting on the main
distribution, but does an amazing job considering the lack of needing to
tune jump size parameters.

    \begin{tcolorbox}[breakable, size=fbox, boxrule=1pt, pad at break*=1mm,colback=cellbackground, colframe=cellborder]
\prompt{In}{incolor}{25}{\boxspacing}
\begin{Verbatim}[commandchars=\\\{\}]
\PY{n}{cplot2}\PY{+w}{ }\PY{o}{=}\PY{+w}{ }\PY{n}{cornerplot}\PY{p}{(}\PY{n}{chain}\PY{p}{,}\PY{+w}{ }\PY{p}{[}\PY{l+s}{\PYZdq{}}\PY{l+s}{M}\PY{l+s}{\PYZdq{}}\PY{p}{,}\PY{+w}{ }\PY{l+s}{\PYZdq{}}\PY{l+s}{Omega\PYZus{}M}\PY{l+s}{\PYZdq{}}\PY{p}{,}\PY{+w}{ }\PY{l+s}{\PYZdq{}}\PY{l+s}{Omega\PYZus{}DE}\PY{l+s}{\PYZdq{}}\PY{p}{]}\PY{p}{)}

\PY{n}{save}\PY{p}{(}\PY{l+s}{\PYZdq{}}\PY{l+s}{AffineMCMCCPlot.png}\PY{l+s}{\PYZdq{}}\PY{p}{,}\PY{+w}{ }\PY{n}{cplot2}\PY{p}{)}
\PY{n}{cplot2}
\end{Verbatim}
\end{tcolorbox}

    \begin{center}
      
      \includegraphics[]{AffineMCMCCPlot.png}
    \end{center}

    
    \begin{Verbatim}[commandchars=\\\{\}]
Bonito.App(Bonito.var"\#8\#14"\{WGLMakie.var"\#20\#21"\{WGLMakie.Screen, Scene\}\}(WGLMakie.var"\#20\#21"\{WGLMakie.Screen, Scene\}(WGLMakie.Screen(
    framerate = 30.0,
    resize\_to = nothing,
    px\_per\_unit = automatic,
    scalefactor = automatic,
    session = nothing,
    three = Channel\{Bool\}(1) (empty),
    scene = Scene (600px, 600px):
  0 Plots
  6 Child Scenes:
    ├ Scene (600px, 600px)
    ├ Scene (600px, 600px)
    ├ Scene (600px, 600px)
    ├ Scene (600px, 600px)
    ├ Scene (600px, 600px)
    └ Scene (600px, 600px),
), Scene (600px, 600px):
  0 Plots
  6 Child Scenes:
    ├ Scene (600px, 600px)
    ├ Scene (600px, 600px)
    ├ Scene (600px, 600px)
    ├ Scene (600px, 600px)
    ├ Scene (600px, 600px)
    └ Scene (600px, 600px))), Base.RefValue\{Union\{Nothing, Bonito.Session\}\}(nothing), "Bonito App", false)
    \end{Verbatim}

    
    Now I plot these contours, pretty smooth :)

    \begin{tcolorbox}[breakable, size=fbox, boxrule=1pt, pad at break*=1mm,colback=cellbackground, colframe=cellborder]
\prompt{In}{incolor}{26}{\boxspacing}
\begin{Verbatim}[commandchars=\\\{\}]
\PY{n}{h}\PY{+w}{ }\PY{o}{=}\PY{+w}{ }\PY{n}{FHist}\PY{o}{.}\PY{n}{Hist2D}\PY{p}{(}\PY{p}{(}\PY{n}{chain}\PY{p}{[}\PY{o}{:}\PY{p}{,}\PY{l+m+mi}{2}\PY{p}{]}\PY{p}{,}\PY{+w}{ }\PY{n}{chain}\PY{p}{[}\PY{o}{:}\PY{p}{,}\PY{l+m+mi}{3}\PY{p}{]}\PY{p}{)}\PY{p}{;}\PY{+w}{ }\PY{n}{nbins}\PY{o}{=}\PY{p}{(}\PY{l+m+mi}{20}\PY{p}{,}\PY{l+m+mi}{20}\PY{p}{)}\PY{p}{)}

\PY{n}{counts}\PY{+w}{ }\PY{o}{=}\PY{+w}{ }\PY{n}{bincounts}\PY{p}{(}\PY{n}{h}\PY{p}{)}\PY{p}{;}
\PY{n}{xc}\PY{p}{,}\PY{n}{yc}\PY{+w}{ }\PY{o}{=}\PY{+w}{ }\PY{n}{bincenters}\PY{p}{(}\PY{n}{h}\PY{p}{)}\PY{p}{;}

\PY{n}{f}\PY{+w}{ }\PY{o}{=}\PY{+w}{ }\PY{n}{Figure}\PY{p}{(}\PY{p}{)}
\PY{n}{Axis}\PY{p}{(}\PY{n}{f}\PY{p}{[}\PY{l+m+mi}{1}\PY{p}{,}\PY{+w}{ }\PY{l+m+mi}{1}\PY{p}{]}\PY{p}{)}

\PY{n}{contour!}\PY{p}{(}\PY{n}{xc}\PY{p}{,}\PY{+w}{ }\PY{n}{yc}\PY{p}{,}\PY{+w}{ }\PY{n}{counts}\PY{p}{)}\PY{c}{\PYZsh{}, levels=[10,50,100])}

\PY{n}{save}\PY{p}{(}\PY{l+s}{\PYZdq{}}\PY{l+s}{AffineMCMCContour.png}\PY{l+s}{\PYZdq{}}\PY{p}{,}\PY{+w}{ }\PY{n}{f}\PY{p}{)}
\PY{n}{f}
\end{Verbatim}
\end{tcolorbox}

    \begin{center}
      \includegraphics[]{AffineMCMCContour.png}
    \end{center}


    
    \begin{Verbatim}[commandchars=\\\{\}]
Bonito.App(Bonito.var"\#8\#14"\{WGLMakie.var"\#20\#21"\{WGLMakie.Screen, Scene\}\}(WGLMakie.var"\#20\#21"\{WGLMakie.Screen, Scene\}(WGLMakie.Screen(
    framerate = 30.0,
    resize\_to = nothing,
    px\_per\_unit = automatic,
    scalefactor = automatic,
    session = nothing,
    three = Channel\{Bool\}(1) (empty),
    scene = Scene (600px, 450px):
  0 Plots
  1 Child Scene:
    └ Scene (600px, 450px),
), Scene (600px, 450px):
  0 Plots
  1 Child Scene:
    └ Scene (600px, 450px))), Base.RefValue\{Union\{Nothing, Bonito.Session\}\}(nothing), "Bonito App", false)
    \end{Verbatim}

    
    This contour looks much smoother now

    \begin{tcolorbox}[breakable, size=fbox, boxrule=1pt, pad at break*=1mm,colback=cellbackground, colframe=cellborder]
\prompt{In}{incolor}{27}{\boxspacing}
\begin{Verbatim}[commandchars=\\\{\}]
\PY{n}{h}\PY{+w}{ }\PY{o}{=}\PY{+w}{ }\PY{n}{FHist}\PY{o}{.}\PY{n}{Hist2D}\PY{p}{(}\PY{p}{(}\PY{n}{chain}\PY{p}{[}\PY{o}{:}\PY{p}{,}\PY{l+m+mi}{2}\PY{p}{]}\PY{p}{,}\PY{+w}{ }\PY{n}{chain}\PY{p}{[}\PY{o}{:}\PY{p}{,}\PY{l+m+mi}{3}\PY{p}{]}\PY{p}{)}\PY{p}{;}\PY{+w}{ }\PY{n}{nbins}\PY{o}{=}\PY{p}{(}\PY{l+m+mi}{100}\PY{p}{,}\PY{l+m+mi}{100}\PY{p}{)}\PY{p}{)}

\PY{n}{counts}\PY{+w}{ }\PY{o}{=}\PY{+w}{ }\PY{n}{bincounts}\PY{p}{(}\PY{n}{h}\PY{p}{)}\PY{p}{;}
\PY{n}{xc}\PY{p}{,}\PY{n}{yc}\PY{+w}{ }\PY{o}{=}\PY{+w}{ }\PY{n}{bincenters}\PY{p}{(}\PY{n}{h}\PY{p}{)}\PY{p}{;}

\PY{n}{f}\PY{+w}{ }\PY{o}{=}\PY{+w}{ }\PY{n}{Figure}\PY{p}{(}\PY{p}{)}
\PY{n}{Axis}\PY{p}{(}\PY{n}{f}\PY{p}{[}\PY{l+m+mi}{1}\PY{p}{,}\PY{+w}{ }\PY{l+m+mi}{1}\PY{p}{]}\PY{p}{)}

\PY{n}{contour!}\PY{p}{(}\PY{n}{xc}\PY{p}{,}\PY{+w}{ }\PY{n}{yc}\PY{p}{,}\PY{+w}{ }\PY{n}{counts}\PY{p}{)}\PY{+w}{ }\PY{c}{\PYZsh{}, levels=[10,50,100])}

\PY{n}{save}\PY{p}{(}\PY{l+s}{\PYZdq{}}\PY{l+s}{finalsmoothcountour.png}\PY{l+s}{\PYZdq{}}\PY{p}{,}\PY{+w}{ }\PY{n}{f}\PY{p}{)}
\PY{n}{f}
\end{Verbatim}
\end{tcolorbox}

    \begin{center}
      \includegraphics[]{finalsmoothcountour.png}
    \end{center}

    
    \begin{Verbatim}[commandchars=\\\{\}]
Bonito.App(Bonito.var"\#8\#14"\{WGLMakie.var"\#20\#21"\{WGLMakie.Screen, Scene\}\}(WGLMakie.var"\#20\#21"\{WGLMakie.Screen, Scene\}(WGLMakie.Screen(
    framerate = 30.0,
    resize\_to = nothing,
    px\_per\_unit = automatic,
    scalefactor = automatic,
    session = nothing,
    three = Channel\{Bool\}(1) (empty),
    scene = Scene (600px, 450px):
  0 Plots
  1 Child Scene:
    └ Scene (600px, 450px),
), Scene (600px, 450px):
  0 Plots
  1 Child Scene:
    └ Scene (600px, 450px))), Base.RefValue\{Union\{Nothing, Bonito.Session\}\}(nothing), "Bonito App", false)
    \end{Verbatim}

    

    % Add a bibliography block to the postdoc
    
    
    
\end{document}
